\documentclass[10pt,a4paper]{beamer}
\usepackage[utf8]{inputenc}
\usepackage[english]{babel}
\usepackage{amsmath}
\usepackage{amsfonts}
\usepackage{amssymb}
\usepackage{placeins}
\usepackage{graphicx}
\author{Benjamin lepers, candidat au BPJEPS kitesurf}
\setbeamertemplate{section in toc}[sections numbered]
\setbeamertemplate{footline}[frame number]

\begin{document}

\title{Initiation et perfectionnement
 kitesurf pour du personnel UCPA à Hourtin}
\begin{frame}
\titlepage
\begin{figure}
\includegraphics[width=4.5cm]{Hourtin_loc1.jpg} 
\end{figure}
\end{frame}

{
\setbeamertemplate{background} 
{
   \includegraphics[width=\paperwidth,height=\paperheight]{Images/spot/background2.jpg}
}
%\setbeamercolor{black}
\setbeamercolor{section in toc}{fg=black}
\setbeamercolor{subsection in toc}{fg=black}
\begin{frame}{Plan}
\tableofcontents
\end{frame}
}

%\subsection{Statut juridique}
%\section{Fonctionnement}
%\section{Zones - Plan}
%\section{Concurrents}


% plus centré sur le projet
% plus expliquer les créneaux, vent NO; thermique
% 

%\section{Introduction}
%lien moi et projet
%{
%\setbeamertemplate{background} 
%{
%   \includegraphics[width=\paperwidth,height=\paperheight]{Images/spot/%background2.jpg}
%}
%\begin{frame}{Introduction}
%\begin{itemize}
%\item quelques mots sur moi: ingénieur de formation, j'ai appris la planche à voile %vers 12 ans
%\item monitorat fédéral de voile en 2003, quelques saisons
%\item vers 2016 premiers stages de kite, puis achat d'une aile, iko en mars 2024
%\item kitesurf un sport pas vraiment accessible 
%\item idée: faire une séance kite à des jeunes en difficulté.
%\end{itemize}
%\end{frame}
%}
\section{UCPA}
\begin{frame}{UCPA}
%\begin{block}*
UCPA (Union Nationale des Centres sportifs de Plein Air): séjours sportifs pour jeunes
et adultes.
%\end{block}
%né de la fusion l'UNCM (Union Nationale des Centres de Montagne)
%et l'UNF (Union Nautique Française) en 1965.

\begin{itemize}
\item permettre aux jeunes et adultes de pratiquer un sport
%\item développer la solidarité et l'autonomie à travers l'activité
%\item favoriser le développement de chacun
\item association à but non lucratif régie par la loi de 1901
%\item organisation de droit privé indépendante de l'Etat.
%\item objectifs d'intér\^et général 
\end{itemize}
\begin{figure}
\includegraphics[width=4cm]{../image_hourtin.jpeg} 
\end{figure}
\end{frame}

\section{Fonctionnement à Hourtin}
\begin{frame}{Fonctionnement à  Hourtin}
\begin{figure}
\includegraphics[width=5cm]{Images/cata.jpg} 
\end{figure}
\begin{itemize}
%\item activités: pav, catamaran, optimist, padle, %kitesurf, wing
\item public jeune en internat (11 à 17 ans), à partir de 5 en externat
\item fonctionnement de type colo, séjour d'une semaine 
\item une séance de kite / jour 13h30-16h30 et 16h30-19h30 
\item 2 moniteurs kite et 2 moniteurs stagiares (Sylvain et Benj)
\item matériel: aile à boudin et hybrides
\end{itemize}
\end{frame}

\section{Organisation}
\begin{frame}{Organisation}
\includegraphics[width=5cm]{zones_kite_voile.jpg} 
%\includegraphics[width=5.5cm]{Images/spot/spot_kite_hourtin2.jpg} 
\includegraphics[width=5.5cm]{Images/ben_stagiairesbateau.jpg} 
\begin{itemize}
\item eau peu profonde
\item vent léger, orientation NO le plus souvent
%\item les ailes hybrides sont beaucoup utilisées.
\item on se rend en bateau sur la zone
\end{itemize}
\end{frame}

\section{Idée initiale}
{
\setbeamertemplate{background} 
{
   \includegraphics[width=\paperwidth,height=\paperheight]{Images/spot/background2.jpg}
}
\begin{frame}{Idée initiale}
\begin{itemize}
\item observation: souvent un public privilégié (catégories socio professionnelles supérieures).
\item  jeunes en difficulté
\item  sortir du quotidien et s'amuser
\item les associations contactées n'ont pas répondu
%\item décision de changer de public en mai avec du %personnel interne à l'UCPA
\end{itemize}
%\begin{figure}
%\includegraphics[width=8cm]{Images/spot/background.jpg} 
%\end{figure}
\begin{figure}
\includegraphics[width=10cm]{Images/projet_planning3.jpg} 
\end{figure}
\end{frame}
}


%\section{Réorientation du projet: changement de public}
%{
%\setbeamertemplate{background} 
%{
%   \includegraphics[width=\paperwidth,height=\paperheight]{Images/spot/%background2.jpg}
%}
%\begin{frame}{Réorientation du projet}
%\begin{itemize}
%\item personnel interne UCPA
%\item personnes sportives et attentives.
%\end{itemize}


%\end{frame}
%}



\section{Diagnostic structure UCPA Hourtin}
%lien projet et structure
%{
%\setbeamertemplate{background} 
%{
%   \includegraphics[width=\paperwidth,height=\paperheight]{Images/spot/background.jpg}
%}
\begin{frame}{Diagnostic structure UCPA Hourtin}
%diagnostic structure UCPA Hourtin\\
%personel ucpa (moniteurs surf, multiactivité, adminstratif et %cuisine)
décision de changer de public en mai avec du personnel interne à l'UCPA

\small{
\begin{table}[h]
\centering
\begin{tabular}{|c|c|}
        \hline
        \textbf{Forces}                          & \textbf{Opportunités} \\ 
        \hline
        plusieurs sports sur le m\^eme site      &outil d'intégration, meilleure ambiance\\
        possibilité d'évoluer dans l'association & plus de satisfaction client  \\
        avantages hébergement et nourriture      &                              \\
               &                             \\
        \hline
        \textbf{Faiblesses}                      &  \textbf{Menaces} \\ 
        \hline
        salaires faibles                         & manque de cohésion d'équipe \\
        horaires décalés                         & concurrence, nouveaux acteurs   \\
        turn over                                &                               \\
        \hline
\end{tabular}
\caption{Analyse SWOT (Strengths, Weakeness, Opportunities, Threat)\label{swot}}
\end{table}}
\end{frame}
%}

\section{Lien projet et structure}
%{
%\setbeamertemplate{background} 
%{
%   \includegraphics[width=\paperwidth,height=\paperheight]{Images/spot/%background.jpg}
%}
\begin{frame}{Lien projet et structure}
\begin{itemize}
\item  centre de kitesurf, bonnes conditions (vent léger, epp, pas de vagues)
\item bénéfices: cohésion des équipes, pratique sportive grisante, spécificités du kite
\item en début de saison:  moniteurs kite peuvent se roder
\item si plus de séances,  postes moniteurs UCPA plus attractifs
\end{itemize}
\end{frame}
%}

\section{Budget}
%\begin{frame}{Budget}
%\begin{figure}
%\includegraphics[width=10cm]{Images/comptes.jpg} 
%\caption{Calcul du prix de rentabilité en tenant compte
%des co\^uts  pour une séance de kite
%avec 3 ailes, 1 bateau, 2 moniteurs et 6 stagiaires}
%\end{figure}
%\end{frame}

%{
%\setbeamertemplate{background} 
%{
%   \includegraphics[width=\paperwidth,height=\paperheight]{Images/spot/%background.jpg}
%}
\begin{frame}{Budget par jour}
\begin{itemize}
\item 10 semaines, 50 jours
\item matériel: $3810 / 50 = 76$
\item personnel: $99$
\item essence: $9$
\item total séance: $184$
\item total par stagiaire (6): $184/6 = 31$
\end{itemize}
\end{frame}
%}


\section{Projet réalisé}
\begin{frame}{Projet réalisé}
\section{Séance initiation}
\begin{itemize}
\item séances initation et perfectionnement au personnel interne UCPA
\item 4 stagiaires, vent léger
\item stagiaires satisfaits de la séance
\item séance perfectionnement, 3 stagiaires, bonne conditions de vent et stagiaires satisfaits
\end{itemize}
\begin{figure}
\includegraphics[width=6cm]{Images/imges_seance_1_2205/hourtin_wind_5.jpg} 
\end{figure}
\end{frame}

\section{Analyse du public}
\begin{frame}{Analyse du public}
\begin{table}
\begin{tabular}{|c|c|c|c|c|}
        \hline
        \textbf{Nom}& \textbf{Age} & \textbf{Poids}& \textbf{Niveau}     &  \textbf{Profession} \\ 
        \hline
       Antony      &  40         &  71           &    Débutant          & Moniteur multiactivité  \\
       \hline
        Tristan       &  35          & 82            &  Débutant           & Moniteur surf  \\
        \hline
        Laurette      &  29          & 54            &  Débutant           & secrétaire \\
        \hline
        Ewan          &  22          & 75            & Débutant            & commis de cuisine  \\
        \hline
         Aurélien      &  34          &  72           &   Perfectionnement  & Moniteur de voile \\
         \hline
        Joé           &  25          &  75           &   Intermédiaire     & Moniteur de voile \\
        \hline
        Rémi          &  25          & 75            &  Intermédiaire      &  Moniteur de surf  \\
        \hline
\end{tabular}
\caption{Analyse du public}
\end{table}
\begin{itemize}
\item  5 moniteurs (voile, multi activité et surf)
\item  personnes sportives, aucune en surpoids
\end{itemize}

\end{frame}


\begin{frame}{Statistiques de vent}
\begin{figure}
\includegraphics[width=\linewidth]{Images/statistics_vent_hourtin.png} 
\end{figure}
\end{frame}
\begin{frame}{Météo typique}
\begin{figure}
\includegraphics[width=\linewidth]{Images/hourtin_wind_220625.jpg} 
\end{figure}
\end{frame}



\begin{frame}{Objectifs des stagiaires}

\begin{table}
\centering
\begin{tabular}{|c|c|c|c|c|}
        \hline
        \textbf{Nom} & \textbf{Age} & \textbf{Poids}& \textbf{Niveau}     &  \textbf{Objectif} \\ 
        \hline
        Antony        &  40          &  71           &   Débutant          & Utiliser la planche  \\
        Tristan       &  35          & 82            &  Débutant           & Tirer des bords  \\
        Laurette      &  29          & 54            &  Débutant           & Prendre du plaisir \\
        Ewan          &  22          & 75            & Débutant            &  -  \\
        \hline
\end{tabular}
\caption{Personnel de l'UCPA pour ma séance initiation kitesurf du 22 mai 2025}
\end{table}
\end{frame}

\begin{frame}{Thèmes séance initiation}
\begin{figure}
\includegraphics[width=8cm]{Images/imges_seance_1_2205/hourtin_wind_2.jpg} 
\end{figure}
\begin{figure}
\item les 3 sécurités (l\^acher, déclencher, libérer)
\item pilotage de l'aile
\item nage tractée 
\end{figure}
\end{frame}

%{
%\setbeamertemplate{background} 
%{
%   \includegraphics[width=\paperwidth,height=\paperheight]{Images/spot/%background.jpg}
%}




\begin{frame}{Thèmes séance initiation}
\begin{table}
\begin{tabular}{|p{4cm}|p{6cm}|}
\hline
\textbf{Exercices}     &  \textbf{Repères}      \\
\hline 
Aile stable à 11h30 ou 13h & Pilotage fin  \\
\hline
Aile stable               & Pilotage à 1 main \\
\hline 
Aile stable               & Pilotage à 1 main en marchant \\
%\hline
%Petits loops  & Action franche d'un coté  \\
\hline 
Redécollage            & bord de fen\^etre, repère de la latte centrale \\
\hline
Huits autour du zenith & Action droite, gauche  \\
\hline
Aile en mouvement, 8      & Nage tractée en descendant le vent \\
\hline 
Aile stable à 10h30 ou 13h30   	& Nage tractée orientée, travers au vent \\
\hline
Aile au zenith, chausser la planche  & Planche perpendiculaire aux lignes, jambes fléchies \\
\hline
Waterstart                           &  Accéleration de l'aile, sortie les fesses de l'eau \\
\hline
\end{tabular}
\caption{Exercices pilotage jusqu'au waterstart, dans le vent très léger, on sera obligé de garder l'aile en mouvement}

\end{table}
\end{frame}
%}
\begin{frame}{Thèmes séance perfectionnement}
\begin{table}
\begin{tabular}{|p{2cm}|p{8cm}|}
\hline
\textbf{Exercices}     &  \textbf{Repères}      \\
\hline 
Bords stable & Pilotage fins, aile stable, regard porté sur la direction   \\
\hline
Bords de près  & Aile stable, regard vers un amer au près, épaules en arrière, crantage  \\
\hline 
Transitions & près, crantage, aile à midi, sustenstation et renvoie de l'aile \\
\hline
Navigation toe side   &  aile stable à 45°, appuis pointe de pieds   \\
\hline
\end{tabular}
\caption{Exercices pour rider, faire du près et demi tour transition}
\end{table}
\end{frame}

%\begin{frame}{Fen\^etre de vol}
%\begin{figure}
%\includegraphics[width=\linewidth]{Images/wind-window-kiteboarding.png} 
%\end{figure}
%\begin{itemize}
%\item zone de puissance, bord de fen\^etre, zenith, huits
%\end{itemize}
%\end{frame}

\begin{frame}{Objectifs des stagiaires}
\begin{table}
\centering
\begin{tabular}{|c|c|c|c|c|}
        \hline
        \textbf{Nom} & \textbf{Age} & \textbf{Poids}& \textbf{Niveau}     &  \textbf{Objectif} \\ 
        \hline
        Aurélien      &  34          &  72           &   Perfectionnement  & Reprise \\
        Joé           &  25          &  75           &   Intermédiaire     & Tirer des bords \\
        Rémi          &  25          & 75            &  Débutant           &  Décoller, premiers sauts  \\
        \hline
\end{tabular}
\caption{Personnel de l'UCPA pour ma séance perfectionnement kitesurf du 2 juin 2025}
\end{table}
\end{frame}



\section{Bilan séance initiation}
%\small
%{
%\setbeamertemplate{background} 
%{
%   \includegraphics[width=\paperwidth,height=\paperheight]{Images/spot/%background.jpg}
%}
\begin{frame}{Bilan}
\begin{table}
\begin{tabular}{|p{1cm}|p{2cm}|p{6cm}|}
        \hline
        \textbf{Nom} & \textbf{Senti en sécurité} & \textbf{Commentaires} \\ 
        \hline
       Antony      &  oui    &   termes techniques plus tard, content d’être en binôme, ludique         \\
       \hline
       Tristan     &  oui    & avoir un cours théorique rapide  \\
       \hline
        Laurette   &  oui    & Un peu trop crispée, mieux avec un kite plus petit \\
        \hline
        Ewan       &  oui    & En pédagogie, plus de métaphores, et expliquer certains termes\\
        \hline
        Aurélien   & oui     & content d'avoir tirer des bords et tenter des sauts  \\
        \hline
        Joé        & oui    &  satisfait de la séance, un plaisir de glisser \\
        \hline
        Rémi       & oui    & content de glisser, quelques tentatives de saut \\
        \hline  
\end{tabular}
\caption{Résumé, retour des stagiaires séance 1}
\end{table}
\begin{itemize}
\item les stagiaires ont appréciés et souhaiteraient en refaire
\item points à améliorer: faire participer plus les stagiaires à la mise en place, 
aile de kite adaptée, explications claires et imagées
\end{itemize}
\end{frame}
%}

\section{Conclusion}
{
\setbeamertemplate{background} 
{
   \includegraphics[width=\paperwidth,height=\paperheight]{Images/spot/background.jpg}
}

\begin{frame}{Conclusion}
\begin{itemize}
\item public content et souhaite continuer
\item bonne idée pour lancer la saison
\item perspectives: plusieurs séances, argument de recrutement 
%\item un regret: ne pas avoir pu toucher le public de jeunes en %difficulté
%\item la saison s'est bien passé, les jeunes commencaient à tirer %des bords en fin de semaine
%\item il n'y a pas eu d'incident en kitesurf, nous étions vigilants au niveau sécurité
%\item merci à mon formateur: Romain Fabretti
\end{itemize}
\end{frame}
% ailes hybrides
}

\end{document}



