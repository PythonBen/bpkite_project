\documentclass[10pt,a4paper]{beamer}
\usepackage[utf8]{inputenc}
\usepackage[english]{babel}
\usepackage{amsmath}
\usepackage{amsfonts}
\usepackage{amssymb}
\usepackage{placeins}
\usepackage{graphicx}
\author{Benjamin lepers, candidat au BPJEPS kitesurf}
\setbeamertemplate{section in toc}[sections numbered]

\begin{document}

\title{Présentation projet: initiation kitesurf pour du personel UCPA}
\begin{frame}
\titlepage
\begin{figure}
\includegraphics[width=4.5cm]{Hourtin_loc1.jpg} 
\end{figure}
\end{frame}

\begin{frame}{Plan}
\tableofcontents
\end{frame}

%\subsection{Statut juridique}
%\section{Fonctionnement}
%\section{Zones - Plan}
%\section{Concurrents}


% plus centré sur le projet
% plus expliquer les créneaux, vent NO; thermique
% 

\section{Introduction}
\begin{frame}{Introduction}
\begin{itemize}
\item quelques mots sur moi: ingénieur de formation, j'ai appris la planche à voile vers 12 ans
\item monitorat fédéral de voile, puis quelques saisons
\item vers 2016 premiers stages de kite, puis achat d'une aile, iko en mars 2024
\item idée: faire une séance kite à des jeunes en difficulté.
\end{itemize}
\end{frame}

\section{UCPA}
\begin{frame}{UCPA}
UCPA (Union Nationale des Centres sportifs de Plein Air) né de la fusion l'UNCM (Union Nationale des Centres de Montagne)
et l'UNF (Union Nautique Française) en 1965.

l'UCPA a pour objectifs:
\begin{itemize}
\item permettre au plus grand nombre jeunes et adultes d'apprendre ou se perfectionner dans un sport
\item développer la solidarité et l'autonomie à travers l'activité
%\item favoriser le développement de chacun
\item association à but non lucratif régie par la loi de 1901
%\item organisation de droit privé indépendante de l'Etat.
%\item objectifs d'intér\^et général 
\end{itemize}
\begin{figure}
\includegraphics[width=4cm]{../image_hourtin.jpeg} 
\end{figure}
\end{frame}

\section{Fonctionnement à Hourtin}
\begin{frame}{Fonctionnement à l'Ucpa Hourtin}
\begin{figure}
\includegraphics[width=4cm]{Images/cata.jpg} 
\end{figure}
\begin{itemize}
%\item activités: pav, catamaran, optimist, padle, %kitesurf, wing
\item public jeune en internat (11 à 17 ans), à partir de 5 en externat
\item fonctionnement de type colo, séjour d'une semaine 
\item une séance de kite / jour 13h30-16h30 et 16h30-19h30 
\item 2 moniteurs kite et 2 moniteurs stagiares (Sylvain et Benj)
\item matériel: aile à boudin et hybrides
\end{itemize}
\end{frame}

\section{Organisation}
\begin{frame}{Organisation}
\includegraphics[width=5cm]{zones_kite_voile.jpg} 
\includegraphics[width=5.5cm]{Images/spot/spot_kite_hourtin2.jpg} 
\begin{itemize}
\item eau peu profonde
\item vent léger, orientation NO le plus souvent
%\item les ailes hybrides sont beaucoup utilisées.
\item on se rend en bateau sur la zone
\end{itemize}
\end{frame}

\section{Idée initiale}
\begin{frame}{Idée initiale}
\begin{itemize}
\item proposer une séance initiation kite à des jeunes en difficulté
\item pour le sortir de leur quotidien et qu'ils s'amusent en sécurité
\item mais les associations contactées n'ont pas répondu
\item décision de changer de public en mai
\end{itemize}
\begin{figure}
\includegraphics[width=8cm]{Images/spot/1000013821.jpg} 
\end{figure}
\end{frame}

\begin{frame}{Statistiques de vent}
\begin{figure}
\includegraphics[width=\linewidth]{Images/statistics_vent_hourtin.png} 
\end{figure}
\end{frame}
\begin{frame}{Météo typique}
\begin{figure}
\includegraphics[width=\linewidth]{Images/hourtin_wind_220625.jpg} 
\end{figure}
\end{frame}

\begin{frame}{Echange avec une association}
\small{
\textbf{Cette annexe contient les échanges de mail que j'ai eu avec l'association 
en gironde Aria33} \\

%Bonjour,

%Je suis actuellement en formation BP Kitesurf à l'école nationale de Quiberon.
%Dans le cadre de notre formation, nous devons réaliser un projet dans
%une structure en lien avec l'enseignement du kitesurf.

%Ma structure d'accueil sera l'UCPA à Hourtin.
%Mon idée de projet est de proposer une journée initiation kitesurf à
%des jeunes qui ont eu un parcours difficile pour leur faire découvrir
%un nouveau sport.
%(pour la tranche d'age, entre 10 et 18 ans par exemple).

%Je me permets de vous contacter pour savoir si il y aurait des
%associations dans la région (proche d'Hourtin) qui pourrait etre
%intéressées par mon projet.

%Je ne sais pas encore exactement le nombre, probablement entre 6 et 8.
%Comme il y a des stages en semaine à l'UCPA, je pensais organiser sur
%le samedi, dimanche ou les deux. Les mois possibles seront mai ou
%juin.

%De mon coté je dois encore vérifier si mon idée est validée par mon
%tuteur. (Je suis en contact).


%Bien cordialement

%Benjamin Lepers

\bigskip
Bonjour
Merci pour cette estimation
En effet, les jeunes que nous accompagnons sont en situation de grande
précarité pour nombre d'entre eux. Si une baisse maximale du prix
pouvait être proposée, nous serions probablement dans la capacité
de constituer un groupe de 6 jeunes. Sur un samedi.

Cordialement
Dolly LEBON
Référente ARIA33

\bigskip
Bonjour,

j'ai contacté ma tutrice de l'Ucpa d'hourtin pour le budget.
Pour une séance de kite  pour 6 élèves, le budget serait  de \textcolor{red} {283 euros, soit 47 euros} par personne.

Si ça bloque peut être qu'elle pourrait baisser un peu.


Je reste disponible.

Cordialement
\bigskip
Bonjour,

j'ai eu le retour de ma responsable de la base d'Hourtin, et l'UCPA
pourrait faire une baisse de \textcolor{red}{20\,\%, soit 226 euros pour 6, soit 37 euros}
par élève.

Cordialement

Benjamin Lepers}
\end{frame}


\section{Changement de public et diagnostic}
\begin{frame}{Changement de public et diagnostic}
diagnostic structure UCPA Hourtin\\
personel ucpa (moniteurs surf, multiactivité, adminstratif et cuisine)
\small{
\begin{table}[h]
\centering
\begin{tabular}{|c|c|}
        \hline
        \textbf{Forces}                          & \textbf{Opportunités} \\ 
        \hline
        plusieurs sports sur le m\^eme site      &outil d'intégration, meilleure ambiance\\
        possibilité d'évoluer dans l'association & plus de satisfaction client  \\
        avantages hébergement et nourriture      &                              \\
        sites nationaux et internationaux        &                             \\
        \hline
        \textbf{Faiblesses}                      &  \textbf{Menaces} \\ 
        \hline
        salaires faibles                         & manque de cohésion d'équipe \\
        horaires décalés                         & concurrence, nouveaux acteurs   \\
        turn over                                &                               \\
        \hline
\end{tabular}
\caption{Analyse SWOT (Strengths, Weakeness, Opportunities, Threat)\label{swot}}
\end{table}}
\end{frame}



\section{Projet réalisé}
\begin{frame}{Projet réalisé}
\begin{itemize}
\item séances initation et perfectionnement au personnel interne UCPA
\item 4 stagiaires, vent léger
\item stagiaires satisfaits de la séance
\item séance perfectionnement, 3 stagiaires, bonne conditions de vent et stagiaires satisfaits
\end{itemize}
\begin{figure}
\includegraphics[width=6cm]{Images/imges_seance_1_2205/hourtin_wind_5.jpg} 
\end{figure}
\end{frame}

\begin{frame}
\small{
\begin{table}
\centering
\begin{tabular}{|c|c|c|c|c|}
        \hline
        \textbf{Nom} & \textbf{Age} & \textbf{Poids}& \textbf{Niveau}     &  \textbf{Objectif} \\ 
        \hline
        Aurélien      &  34          &  72           &   Perfectionnement  & Reprise \\
        Nais          &  25          &  62           &   Débutant          & Comprendre et tirer des bords \\
        Joé         &  25          &  75           &   Intermédiaire     & Tirer des bords \\
        \textcolor{blue}{Antony}        &  40          &  71           &   Débutant          & Utiliser la planche  \\
        Laureleen     &  -           &  -            &   -                 &   -  \\
        \textcolor{blue}{Tristan}       &  35          & 82            &  Débutant           & Tirer des bords  \\
        Rémi          &  25          & 75            &  Perfectionnement           &  Décoller  \\
        \textcolor{blue}{Laurette}      &  29          & 54            &  Débutant           & Prendre du plaisir \\
        \textcolor{blue}{Ewan}          &  22          & 75            & Débutant            &  -  \\
        \hline
\end{tabular}
\caption{Stagiaire de l'Ucpa pour ma séance initiation kitesurf du 22 mai 2025, en bleu mon groupe}
\end{table}}
\end{frame}


\begin{frame}{Thèmes}
\begin{figure}
\includegraphics[width=8cm]{Images/imges_seance_1_2205/hourtin_wind_2.jpg} 
\end{figure}
\begin{figure}
\item les 3 sécurités (l\^acher, déclencher, libérer)
\item pilotage de l'aile
\item nage tractée 
\end{figure}
\end{frame}

\begin{frame}
\small{
\begin{table}
\begin{tabular}{|c|c|c|}
\hline
\textbf{Exercices}     &  \textbf{Repères}      \\
\hline 
Huits autour du zenith & Action droite, gauche   \\
\hline
Petits loops  & Action franche d'un coté  \\
\hline 
Aile stable à 11h30 ou 13h & Pilotage fin \\
\hline
Aile stable               & Pilotage à 1 main \\
\hline 
Aile stable               & Pilotage à 1 main en marchant \\
\hline
Aile en mouvement, 8      & Nage tractée en descendant le vent \\
\hline 
Aile stable à 10h30 ou 13h30   	& Nage tractée orientée, travers au vent \\
\hline
Aile au zenith, chausser la planche  & Planche perpendiculaire aux lignes \\
\hline
Waterstart                           &  Accéleration de l'aile, sortir de l'eau \\
\hline
\end{tabular}
\caption{Exercices pilotage jusqu'au waterstart, dans le vent très léger, on sera obligé de garder l'aile en mouvement}
\end{table}}
\end{frame}

\begin{frame}{Fen\^etre de vol}
\begin{figure}
\includegraphics[width=\linewidth]{Images/wind-window-kiteboarding.png} 
\end{figure}
\begin{itemize}
\item zone de puissance, bord de fen\^etre, zenith, huits
\end{itemize}

\end{frame}

\section{Conclusion}
\begin{frame}{Conclusion}
\begin{itemize}
\item les deux séances de kitesurf à destination du personel interne de l'UCPA: public content et souhaitent continuer
\item bonne idée comme journée d'intégration avec le personnel pour lancer la saison
\item un regret: ne pas avoir pu toucher le public de jeunes en difficulté
\item la saison s'est bien passé, les jeunes commencaient à tirer des bords en fin de semaine
\item il n'y a pas eu d'incident en kitesurf, nous étions vigilants au niveau sécurité
\item merci à mon formateur: Romain Fabretti
\end{itemize}
\end{frame}


%\section{Les ailes hybrides}
%\begin{frame}{Les ailes hybrides}
%\begin{figure}
%\includegraphics[width=5cm]{Images/image_flysufer.jpeg} 
%\end{figure}
%\end{frame}




\end{document}
%\begin{figure}
%\begin{minipage}{0.4\textwidth}
%\includegraphics[width=5cm]{Images/aile-de-kitesurf-flysurfer-hybrid-v2.jpg} 
%\end{minipage}
%\end{figure}
%\begin{figure}
%\includegraphics[width=5cm]{Images/hybride2.jpg} 
%\end{figure}
%\end{frame}
%\begin{itemize}
%\item Conditions de vent: souvent un léger vent thermique, inférieur à 12 noeuds.
%\item Les ailes à boudin volent et redécollent parfois difficilement
%\item Les ailes hybrides sont des ailes à caissons, de type parapente. Elles sont très légères et %volent très bien dans le vent léger.
%\item beaucoup utilisé pour nos cours de kitesurf
%\item principal incovénieut: l'emmelage
%\item $->$ précautionneux pendant le rangement du kite avec sa barre
%\end{itemize}
%\end{frame}

%\section{Mon Formateur: Romain Fabretti}
%\begin{frame}{Mon formateur}
%\begin{itemize}
%\item il a 20 ans d'expérience dans le kite
%\item réparation  et conception des kites
%\item nous a donné de bons conseils %pédagogiques 
%\item formé sur la réparation des kites
%\item le pliage et rangement des hybrides
%\end{itemize}
%\end{frame}

%\section{Saison avec les jeunes}
\begin{frame}{Saison avec les jeunes}
\begin{itemize}
\item jeunes issus d'un milieu social plut\^ot favorisé 
\item bon dialogue avec eux, souvent  premiers bords à la fin de la semaine
\item certains n'étaient pas débutants
\item en moyenne très content des stages de kite
\end{itemize}
\end{frame}

%\section{Conclusion}


%\begin{frame}{SWOT}
%\section{Forces, Faiblesses, Opportunités, Menaces}

%\end{frame}
\end{document}
