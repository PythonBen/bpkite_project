\documentclass[12pt,a4paper]{report}
\usepackage[utf8]{inputenc}
\usepackage[english]{babel}
\usepackage{amsmath}
\usepackage{amsfonts}
\usepackage{amssymb}
\usepackage{graphicx}
\usepackage{geometry}
\usepackage{appendix}
\usepackage{placeins}
\usepackage{pdfpages}
%\documentclass[10pt]{\documentclass[10pt]{\documentclass[10pt]{\documentclass[10pt]{\documentclass[10pt]{•}}}}}\usepackage{biblatex}
\geometry{margin=2.5cm}

\renewcommand{\thesection}{\arabic{section}}
\begin{document}
\author{Benjamin Lepers}
\title{Initiation kitesurf pour du personnel de l'UCPA}
\maketitle
\chapter*{Remerciements}
Je remercie l'UCPA de m'avoir permis de suivre la formation
BPJEPS glisse aérotractées au sein de la structure Ucpa 
d'Hourtin, en particulier Cyrille Pan Wan Sam et d'avoir
mis à ma disposition les moyens pour réaliser mon projet.

Je remercie particulièrement Lucie Poudevigne pour m'avoir
conseillé et aidé dans mon projet.

Je remercie Loic Soufflet et l'équipe des formateurs de 
l'ENVS pour m'avoir formé dans les techniques d'enseignement
du kitesurf.

Je remercie mes compagnons de formation de cette promo 2025,
c'est à dire Etienne, Félix, Léo, Margaux, Marion, Matis, Pattou,
Simon, Stan qui ont rendu cette année très agréable,
avec une très bonne ambiance et entraide entre nous.
Et bien s\^ur Sylvain, mon collègue de formation et 
de structure à Hourtin.

Je remercie également les moniteurs d'Hourtin  pour la
bonne ambiance et le cadre de travail agréable.

Je remercie aussi Romain Frabretti, mon tuteur pour
partager son expérience et tous ses bons conseils.

Enfin, je tiens à remercier mes parents et ma fratrie.


\tableofcontents
%\FloatBarrier
\newpage
\section{Introduction}
J'ai eu la chance d'avoir une famille unie et je n'ai manqué de rien
J'ai pu faire des études et  travailler dans le domaine
scientifique et technique. J'ai pu  faire quelques 
saisons de moniteur voile et de planche à voile entre deux contrat 
de travail.
En 2023, après avoir terminé un projet
en CDD dans le domaine informatique, j'ai passé l'IKO kite\cite{iko}
puis refait une saison en planche à voile et voile à l'UCPA de 
d'Hyères. C'est la que j'ai commencé à réfléchir au BP kite.

C'est surtout l'attrait du plein air, des beaux endroits,
 de l'enseignement, de la rencontre
des gens et un sentiment de liberté avec le   kitesurf qui 
m'ont décidé à m'investir pour cette formation.

Le kitesurf apparaît dans le grand public fin des années 1990 et
l'aile à boudin redécollable est inventée par les frères Legaignoux avec 
un brevet déposé en 1984\cite{brevet_kite}.
Le kitesurf est un sport de sensation et procure un grand
sentiment de liberté. Il est onéreux et n'est pas accessible à tous
financièrement.

Partant de ce constat et effectuant ma partie pratique à l'UCPA\cite{ucpa}, j'ai 
donc décidé de viser un public jeune défavorisé qui n'a  ou n'aura
pas forcément l'opportunité de faire du kitesurf. 
De plus, pour des jeunes en difficulté, ce sport pourra peut être 
leur permettre de voir autre chose, de se déconnecter de leur quotidien,
voir peut être de tomber amoureux de ce sport.

Malheureusement, les associations contactés n'ont pas donné suite
à mes nombreuses sollicitations,
le 9 mai 2025, j'ai donc pris la décision en accord avec 
ma tutrice de réorienter mon projet et de m'adresser au personnel de 
l'UCPA. Ce projet s'est donc transformé en une séance initiation kitesurf 
entre personnel de l'UCPA du centre d'Hourtin.
%\subsection{Contexte et situation géographique}
\section{Présentation de l'UCPA}

L'UCPA ou Union nationale des Centres sportifs de Plein Air, est une association
française à but non lucratif qui a été créée en 1965. Son objectif principal est
de promouvoir l'accès aux activités sportives et de plein air pour le plus grand
nombre, en particulier pour les jeunes et les familles. L'UCPA propose une large
gamme d'activités, allant des sports nautiques comme la voile, le surf et la plongée,
aux sports de montagne comme le ski et l'escalade, en passant par des activités de
pleine nature comme la randonnée et le VTT.

L'association gère des centres de vacances et des bases de loisirs à travers la France,
offrant des séjours sportifs encadrés par des professionnels qualifiés. L'UCPA est
également engagée dans des actions de formation et de sensibilisation à la 
protection de l'environnement et au développement durable.

L’UCPA est agréée entreprise solidaire d’utilité sociale, association de
jeunesse et d’éducation populaire, fédération sportive et partenaire de 
l’éducation nationale. Elle a également les agréments service civique et
vacances adaptées organisées

% to do find statistics from UCPA RH
%\subsection{Historique, objectifs et statut}

L'UCPA est  née de la fusion de deux organisations : l'UNCM 
(Union Nationale des Centres de Montagne)
et l'UNF (Union Nautique Française). Cette fusion a eu lieu en
1965, marquant la création de l'UCPA telle que nous la connaissons aujourd'hui.

L'UNCM et l'UNF étaient toutes deux des associations spécialisées
dans l'organisation de  séjours sportifs, mais chacune avait une
orientation différente : l'UNCM se concentrait sur les activités 
de montagne, tandis que l'UNF se focalisait sur les sports nautiques. 
La fusion de ces deux entités a permis de créer une structure plus 
large et plus diversifiée, capable de proposer une gamme complète 
d'activités sportives et de plein air à un public plus large.

%\subsection{Objectifs}

l'UCPA a pour objectifs de permettre au plus grand nombre jeunes et adultes
d'apprendre ou se perfectionner dans un sport. De développer la solidarité,
l'autonomie à travers l'activité développer la solidarité et l'autonomie
à travers l'activité et le développement personnel de chacun 
(voir appendice \ref{ucpa_projet}).


%\subsection{Statut juridique}

L'UCPA (Union nationale des Centres sportifs de Plein Air) est une association
à but non lucratif régie par la loi de 1901. Ce statut juridique signifie qu'elle est une
organisation de droit privé, indépendante de l'État, et qu'elle poursuit des 
objectifs d'intérêt général sans chercher à réaliser des bénéfices pour ses membres.


Ayant travaillé  deux saisons en 2023 et 2024 à l'UCPA de Hyères pour 
un public jeune et adulte. J'ai  constaté pour le public 
adulte, qu'il est relativement aisé et qualifié. Il est
constitué en majorité de cadres et professions qualifiées. Les
ingénieurs, informaticiens, médecins, avocats, commerciaux et enseignants
représentent une part importante de la clientèle UCPA.
\section{Réorientation du projet}

%Après avoir contacté plusieurs associations en Gironde, 
%je n'ai eu qu'un seul retour. Malheureusement, l'association en question 
%Aria33 n'était pas en mesure de financer la journée.
Début mars, j'ai commencé à contacter des structures d'accueil de jeunes 
en difficulté en gironde. Une structure, l'association Aria33 semblait 
intéressée, mais a bloqué sur le financement (voir appendice \ref{appendix_mail})

J'ai continué mes recherches et contacté trois autres structures par mail et
téléphone. Je n'ai pas eu de retour de mes mails et très peu d'échanges 
au téléphone. 
Devant le peu d’intérêt pour une séance d'initiation kitesurf
des associations contactées, j'ai décidé à regret et en accord avec ma tutrice
de changer de public et de choisir du personnel interne à l'UCPA pour organiser
une séance d'initiation au kitesurf. 
Ceci à l'avantage de régler la question du financement et prévenir des
déconvenues de dernières minutes.

Plusieurs séances ont été programmé et elles adressent en particulier
aux moniteurs autres que le kitesurf, c'est à dire aux moniteurs de voile, kayak et surf, 
ainsi qu'au personnel (administratif et cuisines).
J'ai  proposé 12 places, mon collègue sylvain pourra prendre en charge 6 élèves.
L'intér\^et d'organiser une initiation kitesurf est de faire découvrir ce
sport aux moniteurs et  de favoriser des contacts entre moniteurs dans
une bonne ambiance, des échanges qui n'ont pas forcément lieu pendant la saison.

\section{UCPA Hourtin}
L'UCPA à Hourtin est ma structure d'accueil dans le cadre de ma formation.
\subsection{Organigramme}
\subsection{Type de clientèle}
En internat, le public sont des jeunes 11 à 17 ans, et  à partir de 5 ans  en externat.
En internat, les stagiaires arrivent le dimanche et repartent le samedi suivant.
Certains effectuent des stages de 2 semaines.
\subsection{Fonctionnement}
Il y a rarement du vent le matin, le thermique se lève dans l'après midi.
Le vent est à dominante Ouest, Nord Ouest avec des thermiques légers les après midi et 
en soirée. Les statistiques de vent sont indiquées sur la figure \ref{vent_stats}.
Pour les stagiaires deux séances de kite par jour de 2h45 sont organisées de 13h30 à 16h15
et de 16h30 à 18h45.
Il y a 2 moniteurs kitesurf et 2 stagiaires (Sylvain Lacault) et moi m\^eme.
Une équipe d'environ d'environ 4 moniteurs voiles s'occupent de la voile, 
planche à voile et aussi paddle.

Pour le matériel, la structure dispose d'ailes à boudin et aussi des ailes 
hybrides adaptées au vent léger. En général nous faisons le choix des ailes hybrides 
lorsque le vent moyen est inférieur à 12 noeuds, au delà nous utilisons
les ailes à boudins.

Nous nous rendons en bateau au sud du centre, après le port ou il y a 
une plage (voir carte figure \ref{zones_nav}). Le lac d'Hourtin étant très 
réglementé par la mairie et étant classé zone Natura 2000\cite{natura2000}, 
il y a qu'une assez grande zone délimitée par des bouées et dédié à la pratique
du kitesurf.
Les séances de kitesurf se font principalement en eau peu profonde (~ 1, 1.5m).
La carte de la figure \ref{carte_profondeur} indiquant la profondeur du lac montre que la 
profondeur sur la moitié du lac dans le sens longitudinal 
et inférieur à 2 m. Ceci est du à des raisons géologiques lors de la
formation des dunes du littoral.
%\begin{itemize}
%\item une séance de kite / jour 13h30-16h30 et 16h30-19h30 (thermique en fin d'ap)
%\item 2 moniteurs kite  et 2 moniteurs stagiaires (Sylvain et benj)
%\item une équipe de moniteurs voile
%\item des animateurs (bafa et aussi staps)
%\item matériel: aile à boudin et hybrides (vent < 13 noeuds)
%\item kitesurf: principalement en epp 
%\item Adrien Forestier et Lucie Poudevigne: co directeur village sportif
%\item Lucie Poudevigne: responsable école de sport
%\item tuteur: Romain Fabretti
%\end{itemize}

\subsection{Caractéristiques du lac d'Hourtin}
\begin{figure}
\includegraphics[width=\linewidth]{Images/statistics_vent_hourtin.png} 
\caption{Vent dominant NO, O, assez léger (8-12n), 
souvent thermique en fin d'après midi.\label{vent_stats}}

\end{figure}

\begin{figure}
\includegraphics[width=5cm]{Images/profondeur_hourtin.png} 
\caption{Profondeur du lac d'Hourtin\label{carte_profondeur}}
\end{figure}

\begin{figure}
\includegraphics[width=7cm]{zones_kite_voile.jpg} 
\caption{Zones de navigation, au sud et limité par des bouées jaunes, la zone kite\label{zones_nav}}
\end{figure}
%\begin{figure}
%\includegraphics[width=8cm]{zone_natura_2000.jpg} 
%\caption{Zone natura 2000\ref{zone_natura}}
%\end{figure}

%au sud, zone de kite, moitié du lac $ < 2m$. Enseignement principalement en epp.

\subsection{Zone de naviguation}
\begin{figure}
\begin{minipage}{0.4\textwidth}
\includegraphics[width=5cm]{Images/spot/1000013821.jpg} 
\caption{Zone de pratique}
\end{minipage}
\hfill
\begin{minipage}{0.4\textwidth}
\includegraphics[width=4cm]{Images/spot/1000013820.jpg} 
\caption{Délimitation par des bouées}
\end{minipage}
\end{figure}
La zone de navigation est délimitée par des bouées jaunes.
Sur une grande partie de cette zone, la profondeur varie de 0.5 à 2 m.
Les stagiaires ont donc pied sur presque toute la zone.
\section{Projet}
\subsection{Diagnostic}
%Il y a 3 centres nautiques UCPA dans la région d'Hourtin: Bombannes qui est
%un centre adulte, Hourtin et Montalivet qui sont deux centres à destinations des jeunes.
%Au final, il y a environ 40 ? moniteurs dans les disciplines de la voile, kayak et
%activités terrestres. Tous ces moniteurs ne se fréquentent pas quotidiennement.

L'Ucpa propose des journées d'intégration dans certains centre, mais
une initiation kitesurf est rarement proposée. Les raisons sont notamment
financières avec un risque de casse matériel
et de sécurité avec un risque de blessures.
\subsection{Ressources et contraintes}
\begin{tabular}{|c|c|}
        \hline
        \textbf{Forces}                          & \textbf{Opportunités} \\ 
        \hline
        plusieurs sports sur le m\^eme site      &  outil d'intégration, meilleure ambiance\\
        possibilité d'évoluer dans l'association & plus de satisfaction client  \\
        avantages hébergement et nourriture      &                              \\
        \hline
        \textbf{Faiblesses}                      &  \textbf{Menaces} \\ 
        \hline
        salaires faibles                         & manque de cohésion d'équipe \\
        horaires décalés                         & concurrence, nouveaux acteurs   \\
        \hline
\end{tabular}

\subsection{Budget prévisionnel}
Je suis en contrat pro avec l'UCPA. Le salaire net est de 1396 euros pour 152 heures, 
ce qui fait un salaire horaire net de 9.2 euros.
Il y a Romain notre formateur en surveillance sur un bateau, Sylvain et moi.
Nous comptons une séance de 2h. Le total due aux moniteurs kite est donc de
%2*(9.2+9.2+12) = 40 euros
% to be complete and more detail
\subsection{Proposition}
L'idée d'organiser plusieurs séances de  kitesurf à destination des moniteurs
de ces trois centres permet  de créer des contacts 
entre moniteurs dans une bonne ambiance et en sécurité.

\subsection{Plan d'action}
Plusieurs séances de kitesurf seront proposées aux moniteurs d'Hourtin.
Les dates sont le 22, 23 mai et le 2 et 3 juin.
\subsection{Objectifs}
Cette journée a donc principalement trois buts:\\
- se faire rencontrer des moniteurs qui ne se côtoient pas forcément \\
- s'initier au kitesurf en sécurité \\
- échanger sur l'enseignement pour chaque pratique \\

\subsection{Planification}
Le planning des étapes importantes de mon projet est indiqué
à la figure \ref{gantt}.

\begin{figure}
\centering
\includegraphics[width=12cm]{planningBL.png} 
\caption{Tableau de planification du projet \label{gantt}}
\end{figure}
\subsection{Météo du jour, 22/06/25}
\begin{figure}
\includegraphics[width=\linewidth]{Images/hourtin_wind_220625.jpg} 
\caption{Météo seance 1, du 22/05/25}
\end{figure}
\subsection{Séance 1}
public: Sardine, Aurélien, Nais, Joe, Anthony, laureleen, Tristan, Rémi
themes: pilotage, pilotage 1 main, nage tractée, nage tractée orientée, début waterstart
\begin{table}
\begin{tabular}{|c|c|c|c|c|}
        \hline
        \textbf{Nom} & \textbf{Age} & \textbf{Poids}& \textbf{Niveau}     &  \textbf{Objectif} \\ 
        \hline
        Aurélien      &  34          &  72           &   Perfectionnement  & Reprise \\
        Nais          &  25          &  62           &   Débutant          & Comprendre et tirer des bords \\
        Josué         &  25          &  75           &   Intermédiaire     & Tirer des bords \\
        Antony        &  40          &  71           &   Débutant          & Utiliser la planche  \\
        Laureleen     &  -           &  -            &   -                 &   -  \\
        Tristan       &  35          & 82            &  Débutant           & Tirer des bords  \\
        Rémi          &  25          & 75            &  Débutant           &  Décoller  \\
        Laurette      &  29          & 54            &  Débutant           & Prendre du plaisir \\
        Ewan          &  22          & 75            & Débutant            &  -  \\
        \hline
\end{tabular}
\caption{Stagiaire de l'Ucpa pour ma séance initiation kitesurf du 22 mai 2025}
\end{table}

Remarques: Antony et Tristan: première nage tractée (11 m2)
Laurette: pilotage difficile  (11 m2)
Ewan: bon pilotage  (11m2)

\begin{itemize}
\item Antony: Moniteur multi activité, passé de sportif en rugby. Calme et attentif.
\item Tristan: Moniteur de surf, calme et attentif.
\item Ewan: Travail en cuisine, sportif et à l'écoute
\item Laurette: Travail à l'ucpa en gestion et organisation, attentive.
\end{itemize}
\begin{figure}
\begin{minipage}{0.4\textwidth}
\includegraphics[width=8cm]{Images/imges_seance_1_2205/hourtin_wind_2.jpg} 
\caption{Un stagiaire qui à l'air heureux}
\end{minipage}
\hfill
\begin{minipage}{0.4\textwidth}
\includegraphics[width=8cm]{Images/imges_seance_1_2205/hourtin_wind_3.jpg} 
\caption{Une stagiaire concentrée sur le pilotage}
\end{minipage}
\end{figure}
Le public que j'ai eu pour ma première séance est donc assez sportif
et ils ont tous apprécié la séance. Ils se sont sentis en sécurité
et voudraient bien continuer à progresser si l'opportunité se présentait.
Les réponses de chaque stagiaires sont consignées dans le tableau \ref{questionnaire}.
\subsection{Séance 2}

le 23/05/2025, 
ewan: nage tractée, orientée, premiers waterstart (11 m2)
laureleen: pilotage, nage tractée, (9 m2)


\subsection{Budget prévisionnel}
\subsection{Outils d'évaluation}
\subsection{Bilan des séances}
Le retour des stagiaires, selon le questionnaire de l'annexe \ref{questionnaire}.
\begin{table}
\begin{tabular}{|p{6cm}|p{1.6cm}|p{7cm}|}
        \hline
        \textbf{\small{Questions}}           & \textbf{\small{Stagiaire}} & \textbf{\small{Réponse}}   \\ 
        \hline
  As tu aimé l'activité kitesurf ?           & Ewan                       & Oui c'était super !   \\
        \hline
        Te sentais tu en sécurité ? \\
        / as tu eu peur à un moment ?    &     & \shortstack{Non pas du tout, les consignes de  \\
                                         sécurité ont été bien expliquées, donc\\je me suis senti en total sécurité} \\
        \hline
        As tu compris des choses sur le kitesurf ? & &  \shortstack{Le pilotage et la nage tracté sont  \\    
                                                                    venus assez rapidement mais \\
                                                                     il faudrait encore \\
                                                                    davantage de pratique pour \\
                                                               vraiment assimiler et maitriser \\
                                                                 le tout à 100\%}\\
        \hline
\shortstack{Aimerais tu continuer cette \\ 
         activité si tu en as la  \\
                                     possibilité ?}          & &oui ! \\
         \hline
\shortstack{As des points positifs ou \\
 négatifs  à mentionner ?}                                 & & \shortstack{En terme pédagogique je pense \\
                                                                        qu'il pourrait \^etre bien d'utiliser \\
                                                                        des images ou métaphores pour aider à \\
                                                                        mieux comprendre les différentes \\
                                                                        techniques (exemple du boxeur pour \\
                                                                        maitriser les gauches droites sur le  \\
                                                                        pilotage).Et peut \^etre prendre plus \\
                                                                        de temps sur les termes techniques \\
                                                                        qui sont propre à cette discipline \\
                                                                        (les arrières, la fen\^etre, ect...) \\
                                                                        Sinon, l'activité était super, merci \\
                                                                        beaucoup !}\\
          \hline
          D'autres commentaires ?                       &  &     \\
          \hline                              
       As tu aimé l'activité kitesurf ?      & Antony     & \shortstack{Oui, j'ai beaucoup aimé les sensations\\    du vent  dans la voile et du début de \\ pilotage que nous avons eu dans la voile \\et du début de pilotage que nous \\ avons pu réaliser}\\                                        
       \hline
      Te sentais tu en sécurité ? \\
        / as tu eu peur à un moment ?    &     & \shortstack{Je n'ai pas eu peur et je me suis  \\
                                         senti en sécurité gr\^ace aux conseils\\
                                         et à l'aspect ludique donnée à la séance \\
                                         notamment avec la nage tractée} \\
        \hline  
        As tu compris des choses sur le kitesurf ? & &  \shortstack{J'ai bien compris les premières notions  \\    
                                                                    de pilotage et la logique} \\
        \hline
        \shortstack{Aimerais tu continuer cette \\ 
         activité si tu en as la  \\
                                     possibilité ?}          & & \shortstack{Oui, j'espère avoir la possibilité\\\ d'en refaire afin de pouvoir \\progresser dans l'activité} \\
      \hline
      \shortstack{As des points positifs ou \\
 négatifs  à mentionner ?}                                 & & \shortstack{Peut \^etre aborder les termes \\
                                                                        techniques un peu plus tard ou après \\
                                                                        une définition des termes car j'étais \\
                                                                        un peu perdu. \\
                                                                        L'aspect ludique de la séance et le \\
                                                                        fait que l'on soit en bin\^ome}  \\
      \hline
          D'autres commentaires ?                       &  &  RAS   \\
      \hline
                                                                        
\end{tabular}
\caption{Réponses des stagiaires à mon questionnaire\label{questionnaire}}
\end{table}


\section{Conclusion}

Pendant ma saison à l'UCPA d'Hourtin, j'ai apprécié proposer quelques 
séances d'initiation kitesurf à destination du personnel UCPA.
Les bénéfices ont été:
\begin{itemize}
\item mieux se connaître en faisant une activité plaisante
\item faire progresser des collègues
\item leur faire prendre conscience  des problématiques liées au kitesurf
\end{itemize}

Mes séances se sont bien déroulées en sécurité et en optimisant 
le temps de pratique. Les stagiaires ont été très satisfait de la 
séance et la plupart voudraient continuer l'activité si ils en 
ont la possibilité. 
Au final cette initiation kitesurf s'est révélée un succès.
Il reste un regret, celui de ne pas pu avoir réaliser mon idée initiale, 
à destination d'un public différent. Je pense que la mise en place de séances
pour un tel public sera plus facile dans une structure privée et avec des
contacts direct dans les associations en question.

%\subsection{Matériel}

%\subsection{Périodes de pointe}

%\subsection{Mon tuteur}

%\subsection{Localisation et plan d'accès}



%\subsection{Caractéristiques morphologies}

%\subsection{Aspect psychologique}
%\section{Préparation des séances}
%\subsection{Objectif}
%-s'amuser, rider à la fin de la semaine

%\section{Questionnaire}

%\subsection{Prestations proposées}

%\subsection{Prix et stratégie commerciale}

%\subsection{Concurrents}




%\section{Ressources et contraintes}
%\section{Plan d'action}
%\section{Budget prévisionnel}
%\section{Outils d'évaluation}
%\section{Bilan et conclusion}
%\section{Remerciements}
\appendix
\appendixpage
\addappheadtotoc
\chapter{Questionnaire de satisfaction\label{questionnaire}}
\begin{itemize}
\item As tu aimé l'activité kitesurf ?
\item As tu eu peur à un moment ?
\item Te sentais tu en sécurité ?
\item As tu compris des choses sur le kitesurf ?
%\item Au total était tu content de ton stage ?
\item Aimerais tu continuer cette activité si tu en as la possibilité.
%\item Sur une échelle de 1 à 5 (5 la meilleure note), qu'elle note donnerait tu à ce stage ?
\item As des points positifs ou négatifs a mentionner ?
\item D'autres commentaires ?

\end{itemize}
\chapter{Echanges de mail envoyé à une association Aria33}\label{appendix_mail}
Bonjour,

Je suis actuellement en formation BP Kitesurf à l'école nationale de Quiberon.
Dans le cadre de notre formation, nous devons réaliser un projet dans
une structure en lien avec l'enseignement du kitesurf.

Ma structure d'accueil sera l'UCPA à Hourtin.
Mon idée de projet est de proposer une journée initiation kitesurf à
des jeunes qui ont eu un parcours difficile pour leur faire découvrir
un nouveau sport.
 (pour la tranche d'age, entre 10 et 18 ans par exemple).

Je me permets de vous contacter pour savoir si il y aurait des
associations dans la région (proche d'Hourtin) qui pourrait etre
intéressées par mon projet.

Je ne sais pas encore exactement le nombre, probablement entre 6 et 8.
Comme il y a des stages en semaine à l'UCPA, je pensais organiser sur
le samedi, dimanche ou les deux. Les mois possibles seront mai ou
juin.

De mon coté je dois encore vérifier si mon idée est validée par mon
tuteur. (Je suis en contact).


Bien cordialement

Benjamin Lepers

\bigskip
Bonjour
Merci pour cette estimation
En effet, les jeunes que nous accompagnons sont en situation de grande
précarité pour nombre d'entre eux. Si une baisse maximale du prix
pouvait être proposée, nous serions probablement dans la capacité
de constituer un groupe de 6 jeunes. Sur un samedi.

Cordialement
Dolly LEBON
Référente ARIA33

\bigskip
Bonjour,

j'ai contacté ma tutrice de l'Ucpa d'hourtin pour le budget.
Pour une séance de kite  pour 6 élèves, le budget serait de 283 euros,
soit 47 euros par personnes.

Si ça bloque peut être qu'elle pourrait baisser un peu.


Je reste disponible.

Cordialement
\bigskip
Bonjour,

j'ai eu le retour de ma responsable de la base d'Hourtin, et l'UCPA
pourrait faire une baisse de 20\,\%, soit 226 euros pour 6, soit 37 euros
par élève.

Cordialement

Benjamin Lepers
\chapter{Fiche séance\label{fiche_seance}}
\section{Mise en place}
\begin{itemize}
\item 6 élèves adultes
\item matériel nécessaire: 6 kites, 7,7,9,9,12,12, 3 Twin Tip grandes
\item moniteur: gilet, harnais, coupe ligne, radio, trousse de secours, couteau, coupe ligne
\item bateau: lomac 115 ch
\item météo: 
\end{itemize}
\section{Consignes de sécurité}
\begin{itemize}
\item rappel des 3 sécurités: lâcher, déclencher, libérer
\item distance entre stagiaires: minimum 2 longueurs de lignes
\item si l'aile tombe en feuille morte, on met la main sur le largeur au cas ou elle redécolle et tire
\end{itemize}
\section{Communication}
\begin{itemize}
\item signes: aile au zénith, aile en parking à droite ou gauche, lacher la barre
\item sifflement: on se replace dans la zone
\end{itemize}
\section{déroulé de la séance}
On commence par une aile en l'air pilotée par le moniteur.
Les stagiaires à proximité écoutent les conseils de pilotage et les 3 systèmes
de sécurité.
Ensuite, l'aile est passée à tour de role pour un premier pilotage.
Lorsque, le redécollage est maitrisé et les consignes réspectées, 
on gonfle une 2ème aile, il y a maintenant 3 personnes par kite.
Si ca tourne bien en sécurité, on gonfle un 3ème kite et on a 3 binome par aile.

\chapter{Projet UCPA\label{ucpa_projet}}
\includepdf[angle=90]{projet_ucpa2025.pdf}
\bibliographystyle{plain}
\bibliography{biblio}
\end{document}