\documentclass[11pt,a4paper]{report}
\usepackage[utf8]{inputenc}
\usepackage[frenchb]{babel}
\usepackage{amsmath}
\usepackage{amsfonts}
\usepackage{amssymb}
\usepackage{graphicx}
\usepackage{geometry}
\usepackage{appendix}
\usepackage{placeins}
\usepackage{pdfpages}

%\usepackage{parskip}
%\documentclass[10pt]{\documentclass[10pt]{\documentclass[10pt]{\documentclass[10pt]{\documentclass[10pt]{•}}}}}\usepackage{biblatex}
\geometry{margin=2.5cm}

\renewcommand{\thesection}{\arabic{section}}
%\renewcommand{\contentsname}{Sommaire}
\begin{document}
\author{Benjamin Lepers \\ BPJEPS 2025 }
\title{Initiation kitesurf pour du personnel de l'UCPA Hourtin}
\maketitle
\chapter*{Remerciements}

Je remercie l'UCPA de m'avoir permis de suivre la formation
BPJEPS glisse aérotractées au sein de la structure UCPA
d'Hourtin, et d'avoir mis à ma disposition les moyens nécessaires
pour réaliser mon projet.
 
Je remercie particulièrement Lucie Poudevigne pour m'avoir
conseillé  dans mon projet ainsi que Loic Soufflet et l'équipe des
formateurs de l'ENVSN pour m'avoir formé aux techniques
d'enseignement du kitesurf.

Je remercie mes compagnons de formation de la promotion 2025:
Etienne, Félix, Léo, Margaux, Marion, Matis, Pattou,
Simon, Stan qui ont rendu cette année très agréable,
avec une  bonne ambiance et de l'entraide.
Et bien s\^ur Sylvain, mon collègue de formation et 
de structure à Hourtin.

Je remercie également les moniteurs d'Hourtin  pour la
bonne ambiance et le cadre de travail agréable.

Je suis aussi reconnaissant à  mon tuteur Romain Fabretti
pour m'avoir partagé son expérience et tous ses bons conseils.

Enfin, je pense à  mes parents et ma fratrie.

\tableofcontents
\newpage
\section{Introduction}

J'ai eu la chance de faire plusieurs sports dans ma jeunesse et de pouvoir apprendre la planche à voile pendant mes vacances avec mes grands frères.

Pendant mes études scientifiques, j'ai passé
mon monitorat fédéral de voile. Ensuite, j'ai travaillé dans différents
domaines techniques. Entre deux contrats, j'ai parfois exercé comme 
moniteur de voile pour rester actif et profiter d'un cadre de travail
en extérieur sur l'eau.

%Dans les années 2000, mon frère avait acheter un kite, 
%et nous avons fait quelques tentative ensemble. Mais je trouvai
%le kite impressionnant et je me suis recentré sur la planche à voile.



Plus tard, le kitesurf m'a attiré de plus en plus, 
notamment pour sa plage d'utilisation, sa polyvalence et la 
compacité du matériel. J'ai commencé à m'y intéresser de pr\`es en 
effectuant quelques stages puis en achetant mon matériel.
En 2023, après avoir terminé un projet
dans le domaine informatique, j'ai passé le cours instructeur
IKO\cite{iko} niveau 1 pour augmenter ma connaissance du kitesurf
et de son enseignement en sécurité.

Ensuite en 2023 et 2024, j'ai effectué des saisons complètes de voile
à l'UCPA\cite{ucpa} de Hyères o\`u j'ai pu pratiquer le kitesurf
pendant mon temps libre. L'attrait du plein air, des beaux endroits,
de l'enseignement, de la rencontre des gens et un sentiment de liberté
avec le kitesurf  m'ont décidé à me lancer dans le BPJEPS kitesurf 
à Quiberon.

J'ai  pu obtenir un contrat pro avec l'UCPA et 
ma structure d'accueil est le centre d'Hourtin en Gironde.

\section{Contexte}
\subsection{Historique du kitesurf}

L'origine des premiers cerf-volants remonte à plus de 2000 ans en Chine.
L'idée d'associer un cerf-volant tractant avec une planche est beaucoup
plus récente avec des expérimentations  faites 
dans les années 1970. L'aile à boudin redécollable est inventée 
par les frères Legaignoux avec un brevet déposé en 1984\cite{brevet_kite}.

Le kitesurf se démocratise au début des années 2000 grâce à des
marques comme Naish, Cabrinha et F-One, qui lancent alors la 
production d’ailes destinées au grand public.

Au niveau enseignement, Bruno Legaignoux crée en 1999 le
réseau d'écoles Wipika qui deviendra l'IKO\cite{iko}.
En France, le brevet de moniteur BPJEPS GADA (Glisse Aérotractée 
et Disciplines Associées) est crée en 2003.

\subsection{Présentation de l'UCPA}

L'UCPA ou Union nationale des Centres sportifs de Plein Air, est une association
française créée en 1965. Son objectif principal est
de promouvoir l'accès aux activités sportives et de plein air pour le plus grand
nombre, en particulier pour les jeunes et les familles. Elle propose une large
gamme d'activités, allant des sports nautiques comme la voile, le surf et la plongée,
aux sports de montagne comme le ski et l'escalade, en passant par des activités de
pleine nature comme la randonnée et le VTT.

L'association gère des centres de vacances et des bases de loisirs à travers la
France, offrant des séjours sportifs encadrés par des professionnels qualifiés.
Elle est également engagée dans des actions de formation et de sensibilisation 
à la protection de l'environnement et au développement durable.

Elle est agréée entreprise solidaire d’utilité sociale, association de
jeunesse et d’éducation populaire, fédération sportive et partenaire de 
l’éducation nationale. Elle a également les agréments service civique et
vacances adaptées organisées.

L'UCPA est  née de la fusion de deux organisations: l'UNCM 
(Union Nationale des Centres de Montagne) et l'UNF 
(Union Nautique Française). Cette fusion a eu lieu en 1965, 
marquant la création de l'UCPA telle que nous la connaissons
aujourd'hui.

L'UNCM et l'UNF étaient toutes deux des associations spécialisées
dans l'organisation de séjours sportifs, mais chacune avait une
orientation différente: l'UNCM se concentrait sur les activités 
de montagne, tandis que l'UNF se focalisait sur les sports nautiques. 
La fusion de ces deux entités a permis de créer une structure plus 
large et plus diversifiée, capable de proposer une gamme complète 
d'activités sportives et de plein air à un public plus large.

L'UCPA a pour objectifs de permettre au plus grand nombre
d'apprendre ou de se perfectionner dans un sport.
Pour les sports nautiques, le but est que les stagiaires
tendent vers l'autonomie  et de développer la solidarité entre
pratiquants (voir appendice \ref{ucpa_projet}).

L'UCPA est 
une association à but non lucratif régie par la loi de 1901. 
Ce statut juridique signifie qu'elle est une organisation de
droit privé, indépendante de l'État, et qu'elle poursuit des
objectifs d'intérêt général sans chercher à  réaliser des
bénéfices pour ses membres.

%\FloatBarrier
\subsection{UCPA Hourtin}
\begin{figure}[h]
\centering
\includegraphics[width=16cm]{Images/organigramme.png} 
\caption{Organigramme de l'UCPA Hourtin\label{organi_hourtin}}
\end{figure}

La commune d'Hourtin est située à une soixantaine de km au nord ouest 
de Bordeaux, en Nouvelle Aquitaine. Le lac d'Hourtin est le plus grand
lac d'eau douce de France avec une superficie de $58 \,\,km^2$.
Il s'étend du nord au sud sur une longueur et largeur maximales de 18 km 
par 5 km. Sa profondeur varie de 0 à 7 m, 
d'est en ouest avec une profondeur maximale de 10 m.
Le lac et ses rivages sont des zones naturelles  d'intér\^et écologique, 
faunistique et floristique (zone ZNIEFF type 1 et 2). Il est classé
en zone  Natura 2000\cite{natura2000}, pour protéger la biodiversité
de cet endroit. La figure \ref{lac_znieff} ci-dessous extraite
du site géoportail\cite{geoportail} montre les zones classées znieff 1 et 2.
%\begin{figure}
%\includegraphics[width=5cm]{Images/zone_natura.jpg} 
%\caption{Zone d'intér\^et écologique lac et rivage, zone ZNIEFF 1 et %2\label{lac_znieff}}
%\end{figure}


\begin{figure}
\begin{minipage}{0.4\textwidth}
\includegraphics[width=7cm]{Images/profondeur_hourtin2.png} 
\caption{Carte bathymétrique du lac de Carcans-Hourtin (Université de Bordeaux\cite{bathy}), la profondeur est inférieure
à 2 m sur la moitié est du lac (limite 2 m en vert)\label{carte_profondeur}}
\end{minipage}
\hfill
\begin{minipage}{0.4\textwidth}
%\includegraphics[width=7cm]{zones_kite_voile.jpg} 
\includegraphics[width=8cm]{Images/zone_natura.jpg} 
\caption{Zone d'intér\^et écologique lac et rivage, zone ZNIEFF 1 et 2\label{lac_znieff}}
 \end{minipage}
\end{figure}
\FloatBarrier
Le village sportif UCPA à Hourtin est ma structure d'accueil dans le cadre de
ma formation. Il possède une capacité d'accueil de 118 lits et reçoit un 
public mineur. Le centre fonctionne en été avec 80 collaborateurs pour
proposer du surf, de la voile, du wakeboard, des activités multi sports
et du kitesurf, il est ouvert de mi mai à mi septembre.
 
L'organigramme est donné à la figure \ref{organi_hourtin}.
Le directeur de centre est Adrien Forestier et 
Lucie Poudevigne est la responsable des sports nautiques.
Le centre fonctionne avec l'équipe de moniteurs, le personnel 
de cuisine et de nettoyage. En internat, le public sont des jeunes
11 à 17 ans, et  à partir de 5 ans  en externat.
Les stagiaires arrivent le dimanche et repartent
le samedi suivant. Certains effectuent des séjours de deux semaines.

Il y a rarement du vent le matin, le vent thermique se lève dans
l'après midi, d'orientation dominante Ouest, Nord Ouest avec des
thermiques légers les après midi et  en soirée. Les statistiques
de vent sur l'année sont illustrées à la figure \ref{vent_stats}.
Deux séances de kite par jour de 2h45 sont organisées de
13h30 à 16h15 et de 16h30 à 19h15. 

Pour le produit 100\;\% kite il y a deux groupes de six, les stagiaires ont
une séance par jour et alternent entre le premier créneau et le deuxième.
Pour l'activité kitesurf, il y a deux moniteurs et deux stagiaires
(Sylvain Lacault et moi m\^eme).

Il y a quatre moniteurs pour la voile, planche à voile, paddle et un moniteur 
pour le multi sport (kayak et activités terrestre).

\subsection{Caractéristiques  et zones de navigation du lac d'Hourtin}

La carte de la figure \ref{carte_profondeur} indiquant la 
profondeur du lac montre que la profondeur sur la moitié du lac
dans le sens longitudinal est inférieure à 2 m. 
Ceci est d\^u à des raisons géologiques lors de la
formation des dunes du littoral.
\begin{figure}
\centering
\includegraphics[width=\linewidth]{Images/statistics_vent_hourtin.png} 
\caption{Statistiques de vent à Hourtin lac, vent dominant NO, O, assez léger (8-12n), 
souvent thermique en fin d'après midi.\label{vent_stats}}
\end{figure}

\begin{figure}
\begin{minipage}{0.4\textwidth}
\includegraphics[width=8.5cm]{Images/zone_navig.jpg}
\caption{Zones de navigation, au sud et limité par des bouées jaunes,
la zone kite\label{zones_nav}} 
\end{minipage}
\hfill
\begin{minipage}{0.4\textwidth}
\includegraphics[width=7.5cm]{Images/spot/zone_boues.jpg}
\caption{Délimitation par des bouées\label{boue}}
\end{minipage}
\end{figure}
La zone de navigation est délimitée par des bouées jaunes (voir figure \ref{boue}). 
Sur une grande partie de cette zone, la profondeur varie de 0.5 à 2 m.
Les stagiaires ont donc pied sur presque toute la zone.

\FloatBarrier
\subsection{Matériel}
\begin{figure}[h]
\begin{minipage}{0.4\textwidth}
\includegraphics[width=9cm]{Images/matos2.jpg} 
\caption{Ailes à boudins et ailes hybrides\label{ailes}}
\end{minipage}
\hfill
\begin{minipage}{0.4\textwidth}
\includegraphics[width=6cm]{Images/planches1.jpg} 
\caption{Twin tip de différentes tailles\label{planches}}
\end{minipage}
\end{figure}
Pour le matériel, le centre dispose d'ailes à boudin et aussi 
des ailes hybrides adaptées au vent léger (voir figure \ref{ailes} ).
En général nous faisons
le choix des ailes hybrides lorsque le vent moyen est inférieur à
12 noeuds, au delà nous utilisons les ailes à boudins. Les
surfaces vont de 5 à 15 $m^2$. 
Les ailes à boudins sont des F-One (One et Trust), Slingshot (Z et Ghost).
Les ailes hybrides sont des Flysurfer Hybrid v1 et v2 de surface 6, 7.5, 8,
9.5 et 10 $m^2$. Il y a 5 ailes pour chaque surface.
Les planches sont des F-One, avec des dimensions de $165\times52$ cm
pour la plus grande (figure \ref{planches}).
Les twin tip avec une grande surface favorisent la portance et la glisse
lors des premiers départs. De plus, ces planches larges permettent
de planer plus vite dans le vent léger.

Nous choisissons la surface des kites
en fonction du poids des stagiaires et de la force du vent.
Par exemple pour un adolescent pesant 45 kg dans 15 à 20 noeuds
de vent, pour un premier pilotage, on lui donnera une 6 $m^2$.
Les 7 et 9 $m^2$ à boudin ainsi que  les hybrides sont les ailes
les plus utilisées.
\footnote{L'inconvénient des ailes hybrides est le bridage,
il y a plusieurs rangées de petit filament qui permettent
de donner le profil de l'aile. Il faut donc \^etre très 
vigilant et précautionneux lors du rangement et lors de 
la mise en place, au risque d'avoir des emm\^elages}

\subsection{Organisation des séances}

Nous nous rendons en bateau au sud du centre, après le port o\`u
il y a une plage (voir carte figure \ref{zones_nav}). Le lac 
d'Hourtin est très réglementé par la mairie et est classé
zone Natura 2000\cite{natura2000}. La pratique du kitesurf 
est autorisé dans une  grande zone délimitée par des bouées jaunes. 


Les séances de kitesurf se déroulent principalement en eau peu profonde
entre 0.8 et 1.4 m. L'autre grand avantage de l'endroit est
qu'il n'y a pas de vagues, facilitant ainsi les débuts en pilotage
et waterstart.
Pour des débutants, les premières séances se concentrent sur l'aptitude
à piloter et contrôler son kite. Pour six élèves, on place deux kites en vol.
Au fur et à mesure de la progression dans la ma\^itrise du kite, 
on peut rajouter un kite en l'air pour avoir trois bin\^omes.

\subsubsection{Placements}

En fonction de l'espace latéral à disposition, on place les élèves
alignés avec un minimum de 25 m d'écart entre eux (une longueur de ligne) 
comme indiqué au schéma \ref{kites_alignes}.

Si l'espace latéral est plus restreint, on peut utiliser la configuration
en triangle avec le moniteur au centre du triangle, comme indiqué à la figure
\ref{kite_triangle}.
En pratique et selon les conditions, ces deux configurations se déforment plus ou moins, 
le moniteur explique bien aux stagiaires l'importance de garder une distance minimale 
entre deux binômes. Le moniteur doit garder un oeil en permanence sur le
placement des élèves. On précise aussi la zone avec une bouée sous le vent qui marque
la fin de la zone. Les stagiaires doivent alors replacer l'aile au zénith et 
revenir au point de départ. Parfois on met en place une circulation, 
et le stagiaire travaillant le waterstart va partir du coté opposé si 
un autre stagiaire est sous son  vent. Le respect d'une distance minimale est
très important pour la sécurité.
%\begin{figure}
%\includegraphics[width=6cm]{Images/kite_triangle.png} 
%\caption{Placement des élèves par rapport au moniteur en configuration
%triangle\label{kite_triangle}}
%\end{figure}


\begin{figure}
\begin{minipage}{0.4\textwidth}
\includegraphics[width=6cm]{Images/kites_alignes.png} 
\caption{Placement des élèves par rapport au moniteur dans la configuration
alignés\label{kites_alignes}. Le vent vient du haut}
\end{minipage}
\hfill
\begin{minipage}{0.4\textwidth}
\includegraphics[width=6cm]{Images/kite_triangle.png} 
\caption{Placement des élèves par rapport au moniteur en configuration
triangle\label{kite_triangle}}
\end{minipage}
\end{figure}


%\FloatBarrier




\FloatBarrier
\section{Projet}
\subsection{Diagnostic projet initial}

Le kitesurf est un sport de sensation et procure un grand
sentiment de liberté. Par le pilotage, il oblige à 
la concentration et à \^etre dans l'instant présent.
Mais, c'est un sport onéreux avec de fortes contraintes
au niveau de la sécurité, notamment en raison des lignes du kite.
Pour cette raison, le moniteur doit prendre un groupe réduit
et s'assurer d'un espace suffisant. 

Ayant travaillé deux saisons à l'UCPA de Hyères en voile
et planche à voile, j'ai  constaté que le public adulte est relativement
aisé, constitué en majorité de cadres aux professions qualifiées. Les
ingénieurs, informaticiens, médecins, avocats, commerciaux et enseignants
représentent une part importante de la clientèle UCPA.

Cette observation du marquage social  est corroborée par l'étude sur 
les pratiques sportives de l'enqu\^ete nationale de
2020 menée par le ministère des sports\cite{injep}. Les pratiquants
des sports nautiques, et en particulier du kitesurf sont majoritairement
des profils socio économiques plut\^ot favorisés.

De m\^eme, les jeunes que j'ai pu côtoyer à Hourtin
pendant les séances de kitesurf sont issus de
familles socialement avantagées.

\subsection{Projet initial}

Partant de ce constat et effectuant ma partie pratique à l'UCPA,
j'ai donc décidé de proposer une initiation kitesurf à un public jeune
défavorisé qui n'a ou n'aura pas forcément l'opportunité de faire du kitesurf. 
De plus, pour des jeunes en difficulté, ce sport pourra peut-être 
leur permettre de voir autre chose, de se déconnecter de leur quotidien,
voire avoir un déclic pour ce sport.

Début mars, j'ai commencé à contacter des structures d'accueil de jeunes 
en difficulté en gironde et notamment des MECS (Maisons d'enfants à 
caractère social).
Une structure, l'association Aria33\cite{aria33} semblait 
intéressée, mais a bloqué sur le financement
(voir appendice \ref{appendix_mail}).

J'ai continué mes recherches et contacté trois autres structures
par mail et par téléphone. Je n'ai pas eu de retour de mes mails 
et très peu d'échanges au téléphone.\footnote{Parmi les hypothèses
de non réponses, c'est probablement le manque de financement ou
la réputation du kitesurf comme sport  à risque.}

Les associations contactées ne donnant pas  suite à mes nombreuses
sollicitations. J'ai décidé, à regret et en accord avec ma tutrice
de changer de public le 9 mai 2024 et de choisir du personnel interne
à l'UCPA pour organiser une séance d'initiation et une séance de
perfectionnement au kitesurf.

\subsection{Réorientation du projet\label{reorientation}}

L'UCPA propose des journées d'intégration dans certains centre, mais
une initiation kitesurf est rarement proposée. Les raisons sont notamment
financières avec un risque de casse matériel et de sécurité avec un risque de blessures. Du point de vue du pratiquant, et pour ces m\^emes raisons, ce sport n'est pas facilement  accessible.

%Le kitesurf est un sport de voile spécifique avec des lignes  qui ont
%une longueur autour de  20-25 m. Ainsi, il nécessite de l'espace
%et le moniteur pourra prendre un maximum de quatre ailes en l'air
%pour dispenser un cours de qualité et en sécurité. 

\subsection{Diagnostic de la structure}
\begin{table}[h]
\centering
\begin{tabular}{|c|c|}
        \hline
        \textbf{Forces}                          & \textbf{Opportunités} \\ 
        \hline
        plusieurs sports sur le m\^eme site      &  outil d'intégration, meilleure ambiance\\
        possibilité d'évoluer dans l'association & plus de satisfaction client  \\
        avantages hébergement et nourriture      & fidéliser les moniteurs             \\
        %sites nationaux et internationaux        &                             \\
        \hline
        \textbf{Faiblesses}                      &  \textbf{Menaces} \\ 
        \hline
        salaires faibles                         & manque de cohésion d'équipe \\
        horaires décalées                         & concurrence, nouveaux acteurs   \\
        turn over                                &                               \\
        \hline
\end{tabular}
\caption{Analyse SWOT (Strengths, Weakeness, Opportunities, Threat)\label{swot}}
\end{table}
Le tableau \ref{swot} montre les quelques forces et faiblesses de la 
structure UCPA. Comme c'est une association nationale, il est 
possible pour un  employé motivé de se former et d'évoluer
au sein de l'association, c'est l'une de ses grandes forces.
 
Parmi les faiblesses, pour un moniteur de kite, on peut noter des
salaires plus faibles que dans une structure privée. 
La conséquence directe est qu'il y a un turn over important de moniteurs.
En menace potentielle, il y a  la concurrence avec 
l’apparition de nouveaux acteurs (par exemple décathlon travel),
notamment à l'international. 

\subsection{Autres écoles}

Il y a une école de kite, Hourtin kite school
qui propose des cours pour tout public. 
En pleine saison, une séance de 4h co\^ute 140 euros.
Une autre école qui pratique plus au sud à Lachanau propose
des cours tout public.
Il y a donc une concurrence pour le public jeune mais pas pour le public
adulte, et dans une moindre mesure une concurrence de l'espace de pratique.

\subsection{Proposition}

%Une séance de kitesurf pour le personnel interne de l'UCPA 
%permet de renforcer la cohésion de l'équipe et aussi %d'initier au kitesurf.
%De plus, si il y a plusieurs séances organisées, cela %permettrait de rendre %plus attractif le poste du moniteur %(pour compenser 
%le salaire assez faible).

Au centre d'Hourtin lac, la zone de kite est assez restreinte 
mais bien adaptée à l'apprentissage
du kitesurf avec du vent léger à modéré. De plus nous sommes en 
eau peu profonde avec environ 0.5 à 1 m
de profondeur et sans vagues. Avec ces deux éléments, la zone de kite 
est propice à un apprentissage en sécurité.

%En dehors de cette zone, la pratique de kitesurf sur le lac
%y est interdite. De plus, le lac est en zone Natura %2000\cite{natura2000}, pour
%protéger la biodiversité autour du lac.
%Organiser un parcours downwind ou une régate est donc compliqué
%au niveau réglementaire.

L'idée d'organiser une séance initiation kitesurf
au personnel de l'UCPA d'Hourtin est donc venue naturellement.
Elle permettra au personnel de s'initier au kitesurf en 
sécurité, de participer à une activité plaisante
et de renforcer la cohésion des équipes en cohérence avec le 
projet éducatif et sportif de l'association (voir appendice \ref{ucpa_projet}). 
Plusieurs séances sur la saison permettraient
d'offrir un avantage en nature aux moniteurs et peut-\^etre 
de diminuer le turn over. 

De plus, les moniteurs de voile et de surf pourront se faire une idée des
avantages et inconvénients de ce sport vu sous l'angle de l'enseignement.
Pour les moniteurs de voile en particulier, cet échange permet
de leur montrer les problématiques spécifiques du kite, 
comme la gestion de l'espace et la sécurité liée aux lignes du kite, 
les  systèmes de sécurité 
(lâcher, libérer, larguer), l'espace nécessaire, et la dérive
des stagiaires.

Pour les moniteurs kite, cela leur permettra de se roder en début de
saison sur les cours et de se familiariser avec le matériel.

Plusieurs séances ont été programmées et s'adressent au personnel 
interne de l'UCPA (moniteurs, personnel administratif et
restauration). J'ai  proposé 12 places, mon collègue Sylvain
pourra prendre en charge 6 élèves.
Je présente dans ce rapport la première séance du 22 mai 2025 d'initiation 
au kitesurf et une deuxième séance perfectionnement mise en place
le 2 juin 2025.

\subsection{Rétroplanning}
\begin{figure}[h]
\centering
%\includegraphics[width=12cm]{planningBL.png} 
\includegraphics[width=12cm]{Images/projet_planning3.jpg} 
\caption{Tableau de planification du projet \label{gantt}}
\end{figure}
Le planning des étapes importantes de mon projet est indiqué
à la figure \ref{gantt}. 

En février, j'ai eu l'idée de m'adresser à un public
de jeunes en difficulté, puis contacter les associations aux mois de mars et avril. 
En mai, j'ai décidé  de  réorienter 
le projet avec le changement de public en accord avec
ma responsable. 

J'ai organisé deux séances de kitesurf: une à destination des
débutants le 22 mai et une séance perfectionnement le 2 juin. 


\subsection{Estimation des co\^uts et du seuil de rentabilité}

\begin{figure}
\centering
\includegraphics[width=9cm]{Images/bateaux2.jpg} 
\caption{Semi-rigide utilisés pour les séances de kite\label{bateaux}}
\end{figure}
Pour avoir un ordre d'idée du coût d'une séance initiation kite, je
tiens compte du co\^ut du matériel et du co\^ut salarial. On rapporte 
cela au nombre de journée d'ouverture du centre.

Je suis en contrat pro avec l'UCPA. Le coût pour
l'employeur (brut + charges) est de 2089 euros pour 152 heures par mois, 
ce qui implique un coût horaire de 14 euros pour l'employeur.
Pour mon tuteur, le coût horaire est de 19 euros. Le prix hors taxe
des équipements ainsi que le prix pour la location du moteur des
bateaux est indiqué au tableau \ref{couts}.

\begin{figure}
\includegraphics[width=\linewidth]{Images/comptes.jpg} 
\caption{Calcul du prix de rentabilité en tenant compte
des co\^uts  pour une séance de kite
avec 3 ailes, 1 bateau, 2 moniteurs et 6 stagiaires\label{couts}}
\end{figure}

Le budget prévisionnel permet d'estimer précisément les co\^uts et les recettes.
Ici, je liste tous les coûts impliqués dans la séance d’initiation kite, 
c'est  à dire le matériel avec les kites, les barres, les 
combinaisons ainsi que la location du moteur du bateau. 
Il y a aussi le salaire horaire moniteur. On calcule le coût pour l'UCPA
d'une séance de kite avec trois kites (pour 3 binômes) à la journée.
Le centre de Hourtin est ouvert du 12/05/2025 au 01/09/2025 soit 15 semaines. 
Les stages kitesurf commencent en été, donc en réalité, les kites servent
8 semaines, en comptant mai et juin, on peut arrondir à 10 semaines,
soit 50 jours par an.

Le coût total pour l'employeur pour 3h de séance avec le tuteur moniteur
et l'apprenti moniteur est de $3\times(14+19) = 3\times33 = 99 $.
Le coût total matériel pour 6 personnes est de 3810 euros, soit 76 euros
par jour  en prenant le bateau Myla pour 1 moniteur ($3810/50$).
Le coût de l'essence pour 5 l consommés, est de $1.8\times5 = 9$ euros.
Le coût de la journée est de $99 + 76 + 9 =  184 $ euros.
%Le résumé des coûts journaliers est indiqué au tableau \ref{cout_journalier}.

Ainsi dans ce contexte,  pour six élèves, le seuil de rentabilité
d'un stage serait de $184/6 = 31$ euros
par jour. En dessous, l'UCPA perdrait de l'argent.
On pourrait aussi faire cet exercice de budget prévisionnel avec
l'ensemble du matériel et des moniteurs de kite.

%\begin{table}
%\begin{centering}
%\begin{tabular}{|c|c|c|}
%\hline
%\textbf{Détails}         & \textbf{Dépenses} & \textbf{Recettes}   \\
%\hline
%Matériel  & 76     &                     \\
%\hline                                    
%Essence   & 9      &                     \\
%\hline              
%Moniteurs  & 99    &                     \\
%\hline
%Total     & 184    &       184           \\
%\hline
%\end{tabular}
%\caption{Résumé des co\^uts d'une séance kite à la journée\label{cout_journalier}}
%\end{centering}
%\end{table}

%Cet exercice de lister tous les co\^uts permet de connaître le %seuil 
%de rentabilité et du co\^ut d'un cours de kitesurf 
%dans le centre UCPA d'Hourtin.
\FloatBarrier
\section{Réalisation}
%Plusieurs séances de kitesurf ont été proposées aux moniteurs d'Hourtin.
%Les dates sont le 22, 23 mai et le 2 et 3 juin.
%Je présente ici dans ce projet la séance d'initiation et de perfectionnement.
%Une séance perfectionnement sera organisée début septembre
%et sera présenté dans ce rapport.
Je présente dans ce rapport la séance d'initiation et de perfectionnement
mise en place en avant saison.

A l'UCPA, les groupes sont souvent de six élèves, nous
utilisons donc beaucoup le fonctionnement en bin\^ome. 
Lorsque le stagiaire a une bonne maîtrise de l'aile et
à l'étape du waterstart, nous pouvons commencer à fournir
un kite par élève, en fonction de son niveau d'autonomie.
\subsection{Objectifs: séance 1 22 mai}
\begin{table}[h]
\centering
\begin{tabular}{|c|c|c|c|c|}
        \hline
        \textbf{Nom} & \textbf{Age} & \textbf{Poids}& \textbf{Niveau}     &  \textbf{Objectif} \\ 
        \hline
        Antony        &  40          &  71           &   Débutant          & Utiliser la planche  \\
        Tristan       &  35          & 82            &  Débutant           & Tirer des bords  \\
        Laurette      &  29          & 54            &  Débutant           & Prendre du plaisir \\
        Ewan          &  22          & 75            & Débutant            &  -  \\
        \hline
\end{tabular}
\caption{Personnel de l'UCPA pour ma séance initiation kitesurf du 22 mai 2025\label{stagiaires_table}}
\end{table}
Cette séance a pour objectifs d’initier au pilotage d’un kite et
de comprendre le fonctionnement de la fenêtre de vol. 
Pour les moniteurs de voile, elle vise également à 
sensibiliser aux contraintes spécifiques du kitesurf, 
notamment en matière d’espace et de sécurité.
 
Une après midi entre personnel interne de l'UCPA à travers une 
activité sportive permettra  de mieux se connaître, 
en particulier, cette activité permet de partager en commun
un moment grisant dans un beau cadre naturel. De plus, l'idée
de faire une séance kite est de faire progresser l'élève vers 
l'autonomie (voir le projet ucpa en annexe \ref{ucpa_projet}).

%Enfin, elle participe à la cohésion des équipes
%dans le cadre d'une pratique sportive.

\begin{figure}
\centering
\includegraphics[width=13cm]{Images/hourtin_wind_220625.jpg} 
\caption{Météo seance 1, du 22/05/25\label{meteo}}
\end{figure}

%\subsection{Séance initiation, 22 mai 2025}
%public: Sardine, Aurélien, Nais, Joe, Anthony, laureleen, Tristan, Rémi
%themes: pilotage, pilotage 1 main, nage tractée, nage tractée orientée, début waterstart
\subsection{Analyse du public}
\begin{table}
\begin{tabular}{|c|c|c|c|c|}
        \hline
        \textbf{Nom}& \textbf{Age} & \textbf{Poids}& \textbf{Niveau}     &  \textbf{Profession} \\ 
        \hline
       Antony      &  40         &  71           &    Débutant          & Moniteur multiactivité  \\
       \hline
        Tristan       &  35          & 82            &  Débutant           & Moniteur surf  \\
        \hline
        Laurette      &  29          & 54            &  Débutant           & secrétaire \\
        \hline
        Ewan          &  22          & 75            & Débutant            & commis de cuisine  \\
        \hline
         Aurélien      &  34          &  72           &   Perfectionnement  & Moniteur de voile \\
         \hline
        Joé           &  25          &  75           &   Intermédiaire     & Moniteur de voile \\
        \hline
        Rémi          &  25          & 75            &  Intérmédiaire           &  Moniteur de surf  \\
        \hline
\end{tabular}
\caption{Analyse du public\label{analyse_public}}
\end{table}
Le nom des stagiaires, leur \^age, leur poids, niveau et  leurs objectifs sont
indiqués dans le tableau \ref{stagiaires_table}.
Pour ma séance, je me suis occupé de Antony, Rémi, Laurette et Ewan.
%Sylvain avait Aurélien (autonome), Joé (intermédiaire, tire des bords)
%et Nais (waterstart et premiers bords).
 Le vent était de Nord Ouest entre 
10 et 14 noeuds comme indiqué sur l'image \ref{meteo}.

Il est toujours instructif de connaître ses stagiaires: 
sportifs ou non, antécédents médicaux ou des appréhensions particulières.

Dans mon groupe, il y a Antony, un moniteur multi activité qui a un passé 
de sportif de haut niveau en rugby. Il est calme et attentif.
Tristan est un moniteur de surf, qui lui aussi est calme et attentif.
Ewan qui travaille en cuisine est un jeune sportif à l'écoute, qui avait 
fait déjà un peu de pilotage il y a 2 ans.
Enfin, il y a Laurette qui travaille à l'UCPA en gestion et 
organisation, elle est aussi  sportive et attentive. L'\^age, le poids, la taille
et leurs professions sont données au tableau \ref{analyse_public}

Pour le choix de la taille d'aile, on se base principalement sur la force
de vent, le niveau du rider et son poids. J'avais donc équipé Antony et Tristan
en $11\, m^2$, Ewan qui avait déjà un peu piloté il y a 2 ans, en $11\, m^2$,
 Laurette était aussi en $11\, m^2$.
Avec du recul, j'aurais d\^u lui donner une  $9\, m^2$, elle aurait été un peu plus
à l'aise dans le pilotage. Mais dans les prévisions météo, le vent n'était pas
indiqué à la hausse.

\begin{figure}
\begin{minipage}{0.4\textwidth}
\includegraphics[width=8cm]{Images/imges_seance_1_2205/hourtin_wind_2.jpg} 
\caption{Un stagiaire qui a l'air heureux}
\end{minipage}
\hfill
\begin{minipage}{0.4\textwidth}
\includegraphics[width=8cm]{Images/imges_seance_1_2205/hourtin_wind_3.jpg} 
\caption{Une stagiaire concentrée sur le pilotage, le moniteur tient légèrement
au harnais pour emp\'echer la dérive}
\end{minipage}
\end{figure}

\begin{figure}
\centering
\includegraphics[width=12cm]{Images/imges_seance_1_2205/hourtin_wind_6.jpg}
\caption{Stagiaires en bin\^ome} 
\end{figure}

\begin{table}
\begin{tabular}{|c|c|}
\hline
\textbf{Exercices}     &  \textbf{Repères}      \\
\hline 
Aile stable à 11h30 ou 13h & Pilotage fin  \\
\hline
Redécollage            & bord de fen\^etre, repère de la latte centrale \\
\hline
Aile stable               & Pilotage à 1 main \\
\hline 
Aile stable               & Pilotage à 1 main en marchant \\
%\hline
%Petits loops  & Action franche d'un coté  \\
\hline 
Huits autour du zenith & Action droite, gauche  \\
\hline
Aile en mouvement, 8      & Nage tractée en descendant le vent \\
\hline 
Aile stable à 10h30 ou 13h30   	& Nage tractée orientée, travers au vent \\
\hline
Aile au zenith, chausser la planche  & Planche perpendiculaire aux lignes, jambes fléchies \\
\hline
Waterstart                           &  Accéleration de l'aile, sortie les fesses de l'eau \\
\hline
\end{tabular}
\caption{Exercices pilotage jusqu'au waterstart, dans le vent très léger, on sera obligé de garder l'aile en mouvement\label{seance_pilotage}}
\end{table}

\subsection{Thème de la séance et exercices}


%\subsubsection{Redécollage}
%\begin{figure}[h]
%\centering
%\includegraphics[width=9cm]{Images/ben_stagiaire.jpg}
%\caption{Technique de redecollage d'une aile hybride\label{redecollage}} 
%\end{figure}
%
%Si le vent est léger, on utilisera les ailes hybrides. La technique
%de redécollage dépend si l'aile est à l'endroit ou à l'envers dans l'eau.
%Dans le vent léger, en plein fen\^etre, lorsque
%le kite est à l'endroit, on demande au stagiaire de tirer  sur les avants
%pour faire monter le kite au zénith. 
%Si le kite est à l'envers, on demande au stagiaire de tirer de manière
%égale sur les 2 arrières pour faire monter le kite à 2 m de hauteur en arrière, 
%puis de lâcher une ligne arrière pour qu'il se remette à l'endroit,
%et ensuite reprendre la barre pour le piloter.
%
%Dans le vent plus établi, avec des ailes à boudins, le redécollage s'effectue
%en tirant sur une des arrières, pour le placer en bord de fenêtre.
%Ensuite, lorsque la latte centrale se place à l'horizontale voir légèrement
%vers le haut, le kite est pr\^et à décoller.

Les stagiaires étaient tous des débutants (sauf Ewan, un faux débutant), 
le thème de la séance était d'apprendre le pilotage pour pouvoir contrôler son kite.
Les exercices de pilotage de difficulté croissante visent à une
meilleur maîtrise de l'aile.
Le premier exercice est de stabiliser son kite au zenith, puis à 11h ou 13h.
On montre assez t\^ot la technique de redécollage d'une aile à 
boudin, en tirant sur une arrière pour amener le kite en bord
de fen\^etre et utiliser la latte centrale comme repère (voir annexe \ref{redecollage}).  L'idée est de donner au stagiaire assez vite une certaine
autonomie.
 
Le deuxième exercice est d'effectuer des huits pour explorer la 
fen\^etre de vol.
En fonction de la force du vent, les stagiaires
devaient piloter l'aile plus ou moins dynamiquement. 

Une fois qu'une bonne maitrise du kite est obtenue, 
on peut commencer la  nage tractée avec descente sous 
le vent. Pour cet exercice, il s'agit de se mettre à genoux
et de générer de la puissance avec le kite dans le but
de se faire tracter en avant.  L'idée est de se familiariser
avec le dosage de puissance du kite en le faisant plonger
plus ou moins vite dans la zone de puissance.

Le troisième exercice est un pilotage à une main, il s'agit
de maintenir le kite à une position stable autour
du zenith. On demande aussi au stagiaire de
détacher progressivement son regard de l'aile.
Cet exercice est important pour avoir une main libre
qui servira à prendre la planche ultérieurement.

Le quatrième exercice est la nage tractée orientée.
Le kite doit être placé à 10h30 ou 13h30, le pilotage
est à une main. Le stagiaire se positionne dans l'eau
avec son corps bien gainé en nage indienne. Le bras
libre est bien droit pointant dans une direction 
travers au vent. La nage tractée orientée est assez
technique, mais nécessaire à maîtriser pour récupérer 
sa planche.

Une fois ces étapes validées, on peut
amener une planche et essayer de chausser la planche.
Le but étant de donner la priorité au contrôle du kite, 
puis de chausser la planche et rester en position bien 
groupée pour gagner en stabilité.


L'étape suivante est une sortie du corps de l'eau
en envoyant son aile en zone de puissance et tout en 
gardant les pieds dans les straps. 

L'étape suivante est de partir en glisse avec
une aile stabilisée et une trajectoire constante.

Ma fiche séance est indiquée en annexe \ref{fiche_seance}
et le résumé des exercices de pilotage avec pour
finalité un bon contr\^ole de l'aile est donné
au tableau \ref{seance_pilotage}.

\subsection{Déroulé de la séance}
%Le but de la séance est d'apprendre à faire voler le kite et
%apprendre le pilotage du kite avec sa fen\^etre de vol.
\begin{figure}
\centering
\includegraphics[width=8cm]{Images/wind-window-kiteboarding.png}
\caption{fen\^etre de vol d'un kite\label{fenetre}}
\end{figure} 

Après \^etre arrivé sur zone, nous gonflons le premier kite avec les
lignes préalablement connectées à terre. J'effectue une petite démo
de pilotage, puis fait passer tout le monde chacun son tour pour
un premier pilotage.

Ensuite, je vais passer le kite au binôme Tristan et Antony après leur
avoir expliqué le redécollage. Mon formateur Romain est présent pendant la
séance et il prodigue quelques conseils supplémentaires à Tristan et Antony.
Il me donne aussi quelques astuces et conseils pour le bon déroulé de la séance.

Je gonfle le deuxième kite pour le binome Ewan et Laurette.
Nous restons 20 à 30 min pour travailler les premiers pilotages,
donner des conseils et corriger des erreurs.

Ensuite, voyant que Ewan se débrouillait très bien avec un bon contrôle du 
kite, j'ai décidé de gréer un 3ème kite. Laurette pourra ainsi avoir son kite
et je resterai avec elle.

Il y avait donc pour le reste de la séance, un kite pour Tristan et Antony, 
un kite pour Ewan, et un kite pour Laurette. Je suis resté pas mal de temps
avec Laurette pour la conseiller, car elle avait des difficultés à contrôler l'aile.
J'étais donc le plus souvent avec Laurette et Ewan, pour les corriger techniquement 
et aussi pour effectuer quelques démos.

En fin de séance, j'ai expliqué le waterstart à Ewan et je lui ai
apporté une planche pour qu'il puisse effectuer ses premières sorties
de l'eau avec la planche.

J'ai bien expliqué aux stagiaires que piloter dans le vent léger est
formateur et plus technique que dans du vent établi. La raison est qu'il faut
piloter le kite dynamiquement pour créer plus de vent vitesse pour le kite.
De plus la fen\^etre de vol se réduit, pour que le kite vol dans du vent léger, 
il faut par exemple le faire évoluer entre 11h et 13h. Si le kite s'approche du 
bord de fen\^etre, parfois un loop pour qu'il garde de la vitesse et le replacer
en zone de puissance est la meilleure solution.
Pour les loops dans le vent léger, j'indique aussi qu'il faut
des commandes franches (gros différentiel) pour effectuer un petit loop.
Ce point pour les stagiaires peut sembler contre intuitif.

Au final, la séance s'est bien passé et les stagiaires étaient contents.
Après avoir dégonflé les kites sur le bateaux, nous avons fait un
petit débrief de la séance.
\subsection{Objectifs séance 2: perfectionnement}
%\section{Séance 2 perfectionnement}

\begin{table}[h]
\centering
\begin{tabular}{|c|c|c|c|c|}
        \hline
        \textbf{Nom} & \textbf{Age} & \textbf{Poids}& \textbf{Niveau}     &  \textbf{Objectif} \\ 
        \hline
        Aurélien      &  34          &  72           &   Perfectionnement  & Reprise \\
        Joé           &  25          &  75           &   Intermédiaire     & Tirer des bords \\
        Rémi          &  25          & 75            &  Débutant           &  Décoller, premiers sauts  \\
        \hline
\end{tabular}
\caption{Personnel de l'UCPA pour ma séance perfectionnement kitesurf du 2 juin 2025\label{stagiaires_perf}}
\end{table}
Les objectifs et le nom des stagiaires sont donnés au tableau \ref{stagiaires_perf}.

Dans cette séance, nous avons des stagiaires qui commencent à savoir tirer des bords.
Aurélien est déj\`a un kitesurfeur confirmé, il a pu rider avec une 13 $m^2$, 
remonter au vent et effectuer des petits sauts.
Rémi est un moniteur de surf, il commence à rider et à remonter au vent. 
Il veut apprendre à sauter.
Il a pu tirer des bords et profiter de la séance.
Joé commence aussi à tirer des bords, il a pu travailler le près et les transitions.

%Je me suis placé  sous leur vent, et j'ai pu observer leur bords et %transitions.
Le but de la séance est de passer un bon moment et de progresser en kitesurf.
\subsection{Thèmes de la séance et exercices}
En fonction du niveau, les objectifs sont tirer des bords avec un bon contrôle de la 
trajectoire, remonter au vent, et les transitions. Les repères sont indiqués
au tableau \ref{seance_pilotage2}.

\begin{table}[h]
\begin{tabular}{|c|c|c|}
\hline
\textbf{Exercices}     &  \textbf{Repères}      \\
\hline 
Bords stables & Pilotage fins, aile stable, regard porté sur la direction   \\
\hline
Bords de près  & Aile stable, regard vers un amer au près, épaules en arrière, crantage  \\
\hline 
Transitions & près, crantage, aile à midi, sustenstation et renvoie de l'aile \\
\hline
Navigation toe side   &  aile stable à 45°, appuis pointe de pieds   \\
\hline
\end{tabular}
\caption{Exercices pour rider, faire du près et demi tour transition\label{seance_pilotage2}}
\end{table}

\subsection{Déroulé de la séance}
J'ai préparé une aile de 11 $m^2$ pour Joé et Rémi. Aurélien a pris une aile plus
grande de 13 $m^2$. J'ai indiqué aux stagiaires de bien respecter
la zone entre le bateau et une bouée jaune servant de repère.

Ensuite je me suis placé sous leur vent, et j'ai pu les observer. 
De temps à autre, un stagiaire venait me voir pour des conseils.

Au final, les stagiaires ont pu tirer des bords, et travailler le près. 

\section{Bilan}
Le retour des stagiaires est donné avec le questionnaire de l'annexe \ref{questionnaire}.
Ils ont tous aimé l'activité kitesurf et se sont sentis en sécurité.
Une stagiaire s'est sentie trop crispée et avait peut-\^etre un peu d'appréhension.
Ils  voudraient bien continuer à progresser si l'opportunité se représentait.
Les réponses de chaque stagiaires sont consignées dans l'annexe \ref{reponses}.

Les stagiaires ont aussi compris le fonctionnement d'un kite avec le pilotage.
Ils souhaitent tous en refaire s'ils en ont la possibilité.

En ce qui concerne les points positifs et négatifs, on peut noter
que plus d'explication théorique pour Ewan serait utile, pour Antony
il souhaite moins d'explication des termes techniques, pour plus de clarté.
Pour Tristan, un petit topo théorique avant la mise en place serait utile.

Les axes d'améliorations sont peut-\^etre de faire participer
plus les stagiaires à la mise en place du matériel, et de
donner des consignes simples et claires. 

En résumé, les stagiaires ont apprécié l'initiation kitesurf et
souhaiteraient renouveler l'expérience pour progresser. La table
\ref{stagiaire_feedback} résume les principaux commentaires des stagiaires.

\begin{table}[h]
\begin{tabular}{|c|c|c|}
        \hline
        \textbf{Nom} & \textbf{Senti en sécurité} & \textbf{Commentaires} \\ 
        \hline
       Antony      &  oui    &   termes techniques plus tard, content d’être en binôme, ludique         \\
       \hline
       Tristan     &  oui    & avoir un cours théorique rapide  \\
       \hline
        Laurette   &  oui    & Un peu trop crispée, mieux avec un kite plus petit \\
        \hline
        Ewan       &  oui    & En pédagogie, plus de métaphores, et expliquer certains termes\\
        \hline
\end{tabular}
\caption{Résumé, retour des stagiaires séance 1\label{stagiaire_feedback}}
\end{table}

La deuxième séance pour un public en perfectionnement s'est bien passée et 
mes collègues ont pu rider, remonter au vent et travailler les transitions. Ils étaient tous contents d'y avoir participer.



\FloatBarrier
\section{Conclusion}

Pendant ma saison à l'UCPA d'Hourtin, j'ai pu mettre en place une 
séance d'initiation et de perfectionnement kitesurf à destination 
du personnel interne.
Cette action a permis de renforcer la cohésion d'équipe au sein de l'UCPA
tout en initiant le personnel aux bases du kitesurf dans un cadre 
sécurisé et adapté.

Ma séance s'est bien déroulée en sécurité et en optimisant 
le temps de pratique. Les stagiaires étaient  satisfaits de la 
séance et la plupart voudraient continuer l'activité si il 
y a la possibilité.
Au final cette initiation kitesurf s'est révélée  positive m\^eme si
je manquais encore un peu d'expérience en début de saison.
En ce qui concerne la deuxième séance, le public était déjà confirmé, 
donc je suis moins intervenu directement. Néanmoins, la séance s'est avérée
intéressante et formatrice pour les élèves.

Je pense que proposer une séance initiation kitesurf au 
personnel en début de saison est intéressant pour lancer la saison
et comme moyen de faire connaissance entre les équipes. Le mieux serait de proposer plusieurs séances pendant la saison, cela pourrait aider au recrutement 
des moniteurs à  l'UCPA.

Il reste un regret, celui de ne pas avoir pu réaliser mon idée initiale, 
à destination d'un public moins favorisé. 
Je pense que la clé pour
mettre en place une séance avec tel public est d'avoir un contact
direct et motivé dans une de ces structures.


Cette saison d'été m'a donné plus d'expérience  
de l'enseignement du kite. En particulier dans le vent léger avec les adolescents.
Sur ce point, il faut souligner que les ailes hybrides nous ont beaucoup aidé pour 
faciliter la réussite des élèves. Le revers de la médaille de ces ailes, est qu'il 
faut \^etre précautionneux lors de la mise en place.
Enfin, cette saison m'a fait réfléchir sur les défis du moniteur de kite, en 
particulier du point de vue des conditions météorologiques, il faudra savoir
adapter nos pratiques et notre organisation.

Aparté: Je suis en train d'écrire ce rapport fin juin 2025 et L’Europe subit
une canicule précoce avec des températures de $40^{\circ}$ C en France. Ici à Hourtin, 
il fait $34^{\circ}$ C.
Je m'interroge sur les métiers d'éducateurs plein air dans le futur, il ne sera
probablement plus possible d'exercer en été certains jours
à cause de la chaleur excessive\footnote{
La planète Terre est en train de franchir le seuil  de  $+\, 1.5^{\circ}$ C par rapport
 à l'ère pré industrielle. La consommation de pétrole et de gaz ne baissent pas, les 
émissions de C02 continuent d'augmenter\cite{giec}, un monde avec une anomalie de $+\, 3^{\circ},\, +4^{\circ}$ C  de température moyenne sera beaucoup moins habitable}

\appendix
\appendixpage
\addappheadtotoc

\chapter{\small{Projet UCPA}\label{ucpa_projet}}
%\textbf{Ceci est la brochure de l'UCPA décrivant son projet éducatif
%pour les stagiaires}
%\includepdf[angle=90, width=9cm]{projet_ucpa2025.pdf}[h]
\includepdf[pages=-,fitpaper=true,pagecommand={},angle=90, width=\linewidth]{projet_ucpa2025.pdf}


\chapter{Questionnaire de satisfaction\label{questionnaire}}
\begin{itemize}
\item 1. As tu aimé l'activité kitesurf ?
\item 2. Te sentais tu en sécurité ?, as tu eu peur à un moment ?
\item 3. As tu compris des choses sur le kitesurf ?
\item 4. Aimerais tu continuer cette activité si tu en as la possibilité.
\item 5. As des points positifs ou négatifs a mentionner ?
\item 6. D'autres commentaires ?
\end{itemize}

\chapter{Réponses Stagiaires\label{reponses}}

\begin{itemize}
\item 1. \textbf{Ewan}: Oui, c'était super!, \textbf{Antony}: Oui, j'ai
beaucoup aimé les sensations du vent  dans la voile et du début de
pilotage que nous avons  pu réaliser.                                      
\textbf{Laurette}: Oui, c'était très sympa!, \textbf{Tristan}: Oui,
c'était un moment plaisant avec de bonnes sensations, m\^eme pour
une première séance.
\item 2. \textbf{Ewan}: Non pas du tout, les consignes de sécurité
ont été bien expliquées, donc je me suis senti en total sécurité,
\textbf{Antony}: Je n'ai pas eu peur et je me suis senti en
sécurité gr\^ace aux conseils et à l'aspect ludique donnée à la séance 
notamment avec la nage tractée, \textbf{Laurette}: Je me suis
toujours dit que jamais je ne ferai du kitesurf (peur de m'envoler),
bon c'est ce qui s'est un peu passé! Les explications étaient 
claires et l'activité bien encadrée, je me suis sentie en sécurité.
Par contre, j'étais clairement trop crispée et j'ai peut-\^etre
oublié de respirer. Vive les crampes en fin de séance ! 
\textbf{Tristan}: Je me suis toujours senti en sécurité, le matériel
de protection aide aussi à cela.
\item 3. \textbf{Ewan}: Le pilotage et la nage tracté sont venus
assez rapidement mais il faudrait encore davantage de pratique
pour vraiment assimiler et maîtriser le tout à 100\%
\textbf{Antony}: J'ai bien compris les premières notions de
pilotage et la logique \textbf{Laurette}: Oui j'ai compris
toutes les explications !, les mettre en pratique est un
peu plus compliqué ...
\textbf{Tristan}: J'ai compris quelques bases, je suppose
notamment sur la préparation du matériel et le décollage de l'aile
\item 4. \textbf{Ewan}: Oui!, \textbf{Antony}: Oui, j'espère
avoir la possibilité d'en refaire afin de pouvoir progresser
dans l'activité, \textbf{Laurette}: Ce serait avec grand plaisir
de renouveler l'expérience.
\textbf{Tristan}: Absolument!, 
\item 5. \textbf{Ewan}: En terme pédagogique je pense qu'il
pourrait \^etre bien d'utiliser des images ou métaphores pour
aider à mieux comprendre les différentes techniques (exemple
du boxeur pour maîtriser les gauches droites sur le  pilotage).
Et peut-\^etre prendre plus de temps sur les termes techniques
qui sont propre à cette discipline (les arrières, la fen\^etre, ect...) 
Sinon, l'activité était super, merci beaucoup !
\textbf{Antony}: Peut-\^etre aborder les termes techniques un 
peu plus tard ou après une définition des termes car j'étais un
peu perdu. L'aspect ludique de la séance et le fait que l'on
soit en bin\^ome \textbf{Laurette}: Positifs: moniteurs à
l'écoute, négatifs: j'aurais bien aimé essayer avec une
voile un peu plus adaptée à mon gabarit.\textbf{Tristan}: 
Pas négatif, mais j'imaginerai avoir un cours théorique rapide avant
la mise en pratique afin de comprendre les bases du kitesurf
\item 6 \textbf{Ewan}: ras, \textbf{Antony}: ras, 
\textbf{Laurette}: simplement merci!, \textbf{Tristan}: 
Pas spécialement, la question précédente suffira de mon coté:)
\end{itemize}


\chapter{Echanges de mail envoyé à une association Aria33}\label{appendix_mail}
\textbf{Cette annexe contient les échanges de mail que j'ai eu avec l'association 
en gironde Aria33\cite{aria33}} \\

Bonjour,

Je suis actuellement en formation BP Kitesurf à l'école nationale de Quiberon.
Dans le cadre de notre formation, nous devons réaliser un projet dans
une structure en lien avec l'enseignement du kitesurf.

Ma structure d'accueil sera l'UCPA à Hourtin.
Mon idée de projet est de proposer une journée initiation kitesurf à
des jeunes qui ont eu un parcours difficile pour leur faire découvrir
un nouveau sport.
(pour la tranche d'age, entre 10 et 18 ans par exemple).

Je me permets de vous contacter pour savoir si il y aurait des
associations dans la région (proche d'Hourtin) qui pourrait etre
intéressées par mon projet.

Je ne sais pas encore exactement le nombre, probablement entre 6 et 8.
Comme il y a des stages en semaine à l'UCPA, je pensais organiser sur
le samedi, dimanche ou les deux. Les mois possibles seront mai ou
juin.

De mon coté je dois encore vérifier si mon idée est validée par mon
tuteur. (Je suis en contact).


Bien cordialement

Benjamin Lepers

\bigskip
Bonjour
Merci pour cette estimation
En effet, les jeunes que nous accompagnons sont en situation de grande
précarité pour nombre d'entre eux. Si une baisse maximale du prix
pouvait être proposée, nous serions probablement dans la capacité
de constituer un groupe de 6 jeunes. Sur un samedi.

Cordialement
Dolly LEBON
Référente ARIA33

\bigskip
Bonjour,

j'ai contacté ma tutrice de l'Ucpa d'hourtin pour le budget.
Pour une séance de kite  pour 6 élèves, le budget serait de 283 euros,
soit 47 euros par personnes.

Si ça bloque peut-être qu'elle pourrait baisser un peu.


Je reste disponible.

Cordialement
\bigskip
Bonjour,

j'ai eu le retour de ma responsable de la base d'Hourtin, et l'UCPA
pourrait faire une baisse de 20\,\%, soit 226 euros pour 6, soit 37 euros
par élève.

Cordialement

Benjamin Lepers
\chapter{Fiche séance\label{fiche_seance}}
\section{Mise en place}
\begin{itemize}
\item 6 élèves adultes
\item matériel nécessaire: 6 kites, 7,7,9,9,12,12, 3 Twin Tip grandes
\item moniteur: gilet, harnais, coupe ligne, radio, trousse de secours, couteau, coupe ligne
\item bateau: lomac 115 ch
\item météo: NO, 10-14 noeuds
\end{itemize}
\section{Consignes de sécurité}
\begin{itemize}
\item rappel des 3 sécurités: lâcher, déclencher, libérer
\item distance entre stagiaires: minimum 2 longueurs de lignes
\item si l'aile tombe en feuille morte, on met la main sur le largeur au cas ou elle redécolle et tire
\end{itemize}
\section{Communication}
\begin{itemize}
\item signes: aile au zénith, aile en parking à droite ou gauche, lâcher la barre
\item sifflement: on se replace dans la zone
\end{itemize}
\section{Déroulé de la séance}
On commence par une aile en l'air pilotée par le moniteur.
Les stagiaires à proximité écoutent les conseils de pilotage et les 3 systèmes
de sécurité.
Ensuite, l'aile est passée à tour de rôle pour un premier pilotage.
Lorsque, le redécollage est maîtrisé et les consignes respectées, 
on gonfle une 2ème aile, il y a maintenant trois personnes par kite.
Si le pilotage est bon et les stagiaires tournent bien en sécurité,
on gonfle un 3ème kite et pour avoir finalement trois binômes.

%\chapter{Nombre d'élèves et nombre de kite}
%En fonction du niveau de chaque élève différentes combinaisons 
%sont possible.
%\chapter{\small{Projet UCPA}\label{ucpa_projet}}
%\textbf{Ceci est la brochure de l'UCPA décrivant son projet éducatif
%pour les stagiaires}
%\includepdf[angle=90, width=9cm]{projet_ucpa2025.pdf}[h]
%\includegraphics[angle=90, width=10cm]{projet_ucpa2025.pdf}



\chapter{Technique de redécollage \label{redecollage}}
\section{Redécollage aile hybride}
\begin{figure}[h]
\centering
\includegraphics[width=9cm]{Images/ben_stagiaire.jpg}
\caption{Technique de redecollage d'une aile hybride} 
\end{figure}

Si le vent est léger, on utilisera les ailes hybrides. La technique
de redécollage dépend si l'aile est à l'endroit ou à l'envers dans l'eau.
Dans le vent léger, en plein fen\^etre, lorsque
le kite est à l'endroit, on demande au stagiaire de tirer  sur les avants
pour faire monter le kite au zénith. 
Si le kite est à l'envers, on demande au stagiaire de tirer de manière
égale sur les 2 arrières pour faire monter le kite à 2 m de hauteur en arrière, 
puis de lâcher une ligne arrière pour qu'il se remette à l'endroit,
et ensuite reprendre la barre pour le piloter.

\section{Redécollage aile à boudin}
Dans le vent plus établi, avec des ailes à boudins, le redécollage s'effectue
en tirant sur une des arrières, pour le placer en bord de fenêtre.
Ensuite, lorsque la latte centrale se place à l'horizontale voir légèrement
vers le haut, le kite est pr\^et à décoller.
%\FloatBarrier
%\chapter{\small{Projet UCPA}\label{ucpa_projet}}
%\textbf{Ceci est la brochure de l'UCPA décrivant son projet éducatif
%pour les stagiaires}
%\includepdf[angle=90, width=9cm]{projet_ucpa2025.pdf}[h]
%\includegraphics[angle=90, width=10cm]{projet_ucpa2025.pdf}
\bibliographystyle{plain}
\bibliography{biblio}
\end{document}