\documentclass[11pt,a4paper]{report}
\usepackage[utf8]{inputenc}
\usepackage[frenchb]{babel}
\usepackage{amsmath}
\usepackage{amsfonts}
\usepackage{amssymb}
\usepackage{graphicx}
\usepackage{geometry}
\usepackage{appendix}
\usepackage{placeins}
\usepackage{pdfpages}

%\usepackage{parskip}
%\documentclass[10pt]{\documentclass[10pt]{\documentclass[10pt]{\documentclass[10pt]{\documentclass[10pt]{•}}}}}\usepackage{biblatex}
\geometry{margin=2.5cm}

\renewcommand{\thesection}{\arabic{section}}
%\renewcommand{\contentsname}{Sommaire}
\begin{document}
\author{Benjamin Lepers \\ BPJEPS 2025 }
\title{Initiation kitesurf pour du personnel de l'UCPA}
\maketitle
\chapter*{Remerciements}
Je remercie l'UCPA de m'avoir permis de suivre la formation
BPJEPS glisse aérotractées au sein de la structure Ucpa 
d'Hourtin et d'avoir
mis à ma disposition les moyens pour réaliser mon projet.
 
Je remercie particulièrement Lucie Poudevigne pour m'avoir
conseillé  dans mon projet et Loic Soufflet ainsi que 
l'équipe des formateurs de l'ENVS pour m'avoir formé 
dans les techniques d'enseignement du kitesurf.

Je remercie mes compagnons de formation de cette promo 2025,
c'est à dire Etienne, Félix, Léo, Margaux, Marion, Matis, Pattou,
Simon, Stan qui ont rendu cette année très agréable,
avec une très bonne ambiance et entraide entre nous.
Et bien s\^ur Sylvain, mon collègue de formation et 
de structure à Hourtin.

Je remercie également les moniteurs d'Hourtin  pour la
bonne ambiance et le cadre de travail agréable.

Je remercie aussi Romain Fabretti, mon tuteur pour
partager son expérience et tous ses bons conseils.

Enfin, je pense à  mes parents et ma fratrie.

\tableofcontents
\newpage
\section{Introduction}

J'ai suivi des études scientifiques qui m'intéressaient, puis
travaillé dans différents domaines techniques. Entre deux contrats en 
France et à l'étranger, j'ai parfois exercé comme 
moniteur de voile   pour rester
actif et profiter d'un cadre de travail en extérieur sur l'eau.

Comme pratique nautique, le kite m'a attiré de plus en plus, 
notamment pour sa plage d'utilisation, sa polyvalence 
et la compacité du matériel.
En 2023, après avoir terminé un projet dans le domaine informatique,
j'ai passé le cours instructeur IKO\cite{iko} niveau 1
pour augmenter ma connaissance du kitesurf et de son enseignement
en sécurité.

Ensuite, j'ai refait une saison complète de voile à l'UCPA\cite{ucpa}
de Hyères. L'attrait du plein air, des beaux endroits,
de l'enseignement, de la rencontre
des gens et un sentiment de liberté avec le kitesurf  
m'ont décidé à me lancer dans le BPJEPS kitesurf à 
Quiberon.

J'ai  pu obtenir un contrat pro avec l'UCPA et 
ma structure d'accueil est l'UCPA à Hourtin.

\section{Contexte}
\subsection{Historique du kitesurf}
Des expérimentations ont été faites avec des cerf volants tractants 
dans les années 1970. 
L'aile à boudin redécollable est inventée par les frères Legaignoux avec 
un brevet déposé en 1984\cite{brevet_kite}.

Le kitesurf se démocratise au début des années 2000 grâce à des
marques comme Naish, Cabrinha et F-One, qui lancent alors la 
production d’ailes destinées au grand public.

Au niveau enseignement Bruno Legaignoux crée en 1999 le
réseau d'écoles Wipika qui deviendra l'IKO\cite{iko}.
En France, le brevet de moniteur BPJEPS GADA (Glisse Aérotractée 
et Disciplines Associées) est crée en 2003.

\subsection{Présentation de l'UCPA}

L'UCPA ou Union nationale des Centres sportifs de Plein Air, est une association
française à but non lucratif qui a été créée en 1965. Son objectif principal est
de promouvoir l'accès aux activités sportives et de plein air pour le plus grand
nombre, en particulier pour les jeunes et les familles. L'UCPA propose une large
gamme d'activités, allant des sports nautiques comme la voile, le surf et la plongée,
aux sports de montagne comme le ski et l'escalade, en passant par des activités de
pleine nature comme la randonnée et le VTT.

L'association gère des centres de vacances et des bases de loisirs à travers la
France, offrant des séjours sportifs encadrés par des professionnels qualifiés.
L'UCPA est également engagée dans des actions de formation et de sensibilisation 
à la protection de l'environnement et au développement durable.

L’UCPA est agrée entreprise solidaire d’utilité sociale, association de
jeunesse et d’éducation populaire, fédération sportive et partenaire de 
l’éducation nationale. Elle a également les agréments service civique et
vacances adaptées organisées

L'UCPA est  née de la fusion de deux organisations : l'UNCM 
(Union Nationale des Centres de Montagne)
et l'UNF (Union Nautique Française). Cette fusion a eu lieu en
1965, marquant la création de l'UCPA telle que nous la connaissons
aujourd'hui.

L'UNCM et l'UNF étaient toutes deux des associations spécialisées
dans l'organisation de séjours sportifs, mais chacune avait une
orientation différente: l'UNCM se concentrait sur les activités 
de montagne, tandis que l'UNF se focalisait sur les sports nautiques. 
La fusion de ces deux entités a permis de créer une structure plus 
large et plus diversifiée, capable de proposer une gamme complète 
d'activités sportives et de plein air à un public plus large.

L'UCPA a pour objectifs de permettre au plus grand nombre
d'apprendre ou de se perfectionner dans un sport.
Pour les sports nautiques, le but est que les stagiaires
tendent vers l'autonomie  et de développer la solidarité entre pratiquants.
(voir appendice \ref{ucpa_projet}).


L'UCPA (Union nationale des Centres sportifs de Plein Air) est une association
à but non lucratif régie par la loi de 1901. Ce statut juridique signifie
qu'elle est une organisation de droit privé, indépendante de l'État, et 
qu'elle poursuit des objectifs d'intérêt général sans chercher à 
réaliser des bénéfices pour ses membres.

\subsection{UCPA Hourtin}

La commune d'Hourtin est située à une soixantaine de km au nord ouest 
de Bordeaux, en Nouvelle Aquitaine. Le lac d'Hourtin est le plus grand
lac d'eau douce de France avec une superficie de $58 \,\,km^2$.
Il est classé en zone Natura 2000\cite{natura2000}, 
pour protéger la biodiversité de cet endroit.

Le village sportif UCPA à Hourtin est ma structure d'accueil dans le cadre de
ma formation. Il possède une capacité d'accueil de 118 lits et reçoit un 
public mineur. Le centre fonctionne en été avec 80 collaborateurs pour
proposer du surf, de la voile, du wakeboard, des activités multi sports
et du kitesurf.
Le centre est ouvert de mi mai à mi septembre.
 
\begin{figure}[h]
\centering
\includegraphics[width=10cm]{Images/organigramme_hourtin.png} 
\caption{Organigramme de l'UCPA Hourtin\label{organi_hourtin}}
\end{figure}
L'organigramme est donné à la figure \ref{organi_hourtin}.
Le directeur de centre est Adrien Forestier et 
Lucie Poudevigne est la responsable des sports nautiques.
Le centre fonctionne avec
l'équipe de moniteurs, le personnel de cuisine et de nettoyage.
En internat, le public sont des jeunes 11 à 17 ans, et  à partir 
de 5 ans  en externat.
En internat, les stagiaires arrivent le dimanche et repartent
le samedi suivant. Certains effectuent des stages de 2 semaines.

Il y a rarement du vent le matin, le vent thermique se lève dans
l'après midi, d'orientation dominante Ouest, Nord Ouest avec des
thermiques légers les après midi et  en soirée. Les statistiques
de vent sur l'année sont illustrées à la figure \ref{vent_stats}.
Deux séances de kite par jour de 2h45 sont organisées de
13h30 à 16h15 et de 16h30 à 19h15. 

En 100\% kite il y a deux groupes de 6, les stagiaires ont une
séance par jour et alternent entre le premier créneau et le deuxième.
Il y a 2 moniteurs kitesurf et 2 stagiaires (Sylvain Lacault et moi m\^eme).
En voile, planche à voile et paddle, il y a 3 moniteurs avec un stagiaire.
En multisport, il y a un moniteur et 1 stagiaire.

\subsection{Matériel}
Pour le matériel, la structure dispose d'ailes à boudin et aussi 
des ailes hybrides adaptées au vent léger. En général nous faisons
le choix des ailes hybrides lorsque le vent moyen est inférieur à
12 noeuds, au delà nous utilisons les ailes à boudins. Les
surfaces vont de 5 à 15 $m^2$.

Nous choisissons la surface des kites
en fonction du poids des stagiaires et de la force du vent.
Par exemple pour un adolescent pesant 45 kg dans 15 à 20 noeuds
de vent, pour un premier pilotage, on lui donnera une 6 $m^2$.

En planche, nous utilisons des twin tip de grande dimensions 
pour faciliter la portance lors des premiers départ dans l'eau.
De plus, ces planches larges permettent de planer plus vite
dans le vent léger.

\subsection{Organisation des séances}
Nous nous rendons en bateau au sud du centre, après le port o\`u il y a 
une plage (voir carte figure \ref{zones_nav}). Le lac d'Hourtin étant très 
réglementé par la mairie et étant classé zone Natura 2000\cite{natura2000}, 
il y a qu'une assez grande zone délimitée par des bouées et dédiée à la pratique
du kitesurf.
Les séances de kitesurf se font principalement en eau peu profonde entre 0.8 et 1.4 m.
L'autre grand avantage de l'endroit est qu'il y a pas de vagues facilitant ainsi
les débuts en kite.
La carte de la figure \ref{carte_profondeur} indiquant la profondeur du lac
montre que la profondeur sur la moitié du lac dans le sens longitudinal 
et inférieure à 2 m. Ceci est du à des raisons géologiques lors de la
formation des dunes du littoral.

\subsection{Caractéristiques  et zones de naviguation du lac d'Hourtin}
\begin{figure}
\centering
\includegraphics[width=\linewidth]{Images/statistics_vent_hourtin.png} 
\caption{Vent dominant NO, O, assez léger (8-12n), 
souvent thermique en fin d'après midi.\label{vent_stats}}
\end{figure}

\begin{figure}
\begin{minipage}{0.4\textwidth}
\includegraphics[width=6cm]{Images/profondeur_hourtin.png} 
\caption{Profondeur du lac d'Hourtin, la profondeur est inférieure
à 2m sur la moitié est du lac (limite 2 m en vert)\label{carte_profondeur}}
\end{minipage}
\hfill
\begin{minipage}{0.4\textwidth}
%\includegraphics[width=7cm]{zones_kite_voile.jpg} 
\includegraphics[width=8cm]{Images/zone_navig.jpg} 
\caption{Zones de navigation, au sud et limité par des bouées jaunes,
 la zone kite\label{zones_nav}}
 \end{minipage}
\end{figure}

\begin{figure}
\begin{minipage}{0.4\textwidth}
\includegraphics[width=5cm]{Images/spot/1000013821.jpg} 
\caption{Zone de pratique}
\end{minipage}
\hfill
\begin{minipage}{0.4\textwidth}
\includegraphics[width=4cm]{Images/spot/1000013820.jpg} 
\caption{Délimitation par des bouées}
\end{minipage}
\end{figure}
La zone de navigation est délimitée par des bouées jaunes.
Sur une grande partie de cette zone, la profondeur varie de 0.5 à 2 m.
Les stagiaires ont donc pied sur presque toute la zone.

\FloatBarrier
\section{Projet}
\subsection{Diagnostic projet initial}

Le kitesurf est un sport de sensation et procure un grand
sentiment de liberté. Par le pilotage, il oblige à 
la concentration et à \^etre dans l'instant présent.
Mais, c'est un sport onéreux avec de fortes contraintes
au niveau de la sécurité, notamment en raison des lignes du kite.
Pour cette raison, le moniteur doit prendre un groupe réduit
et s'assurer d'une zone  d'espace suffisant. 

Ayant travaillé deux saisons en 2023 et 2024 à l'UCPA de Hyères en voile/pav
pour un public jeune et adulte, j'ai  constaté que le public 
adulte est relativement aisé et qualifié. Il est
constitué en majorité de cadres et professions qualifiées. Les
ingénieurs, informaticiens, médecins, avocats, commerciaux et enseignants
représentent une part importante de la clientèle UCPA.
Cette observation du marquage social  est corroborée par l'étude sur 
les pratiques sportives de l'enqu\^ete nationale de
2020 menée par le ministère des sports\cite{injep}. Les pratiquants
des sports nautiques, et en particulier du kitesurf sont majoritairement
des profils socio économiques plut\^ot favorisés.

De m\^emes, les jeunes que j'ai pu côtoyer à Hourtin
pendant les séances de kitesurf sont  issus de
familles socialement avantagées.

\subsection{Projet initial}
Partant de ce constat et effectuant ma partie pratique à l'UCPA,
j'ai donc décidé de proposer une initiation kitesurf à  un public jeune
défavorisé qui n'a ou n'aura pas forcément l'opportunité de faire du kitesurf. 
De plus, pour des jeunes en difficulté, ce sport pourra peut être 
leur permettre de voir autre chose, de se déconnecter de leur quotidien,
voir de tomber amoureux de ce sport.

Début mars, j'ai commencé à contacter des structures d'accueil de jeunes 
en difficulté en gironde et notamment des MECS(Maisons d'enfants à caractère social).
 Une structure, l'association Aria33\cite{aria33} semblait 
intéressée, mais a bloqué sur le financement (voir appendice \ref{appendix_mail}).

J'ai continué mes recherches et contacté trois autres structures par mail et
téléphone. Je n'ai pas eu de retour de mes mails et très peu d'échanges 
au téléphone.\footnote{Parmis les hypothèses de non réponses, c'est probablement 
le manque de financement et peut \^etre la réputation du kitesurf comme sport extr\^eme 
ou à risque.}

Les associations contactées ne donnant pas  suite
à mes nombreuses sollicitations, j'ai décidé à regret et en accord avec ma tutrice
de changer de public le 9 mai 2024 et de choisir du personnel interne à l'UCPA pour organiser une séance d'initiation au kitesurf.

\subsection{Réorientation du projet\label{reorientation}}

L'Ucpa propose des journées d'intégration dans certains centre, mais
une initiation kitesurf est rarement proposée. Les raisons sont notamment
financières avec un risque de casse matériel et de sécurité avec un risque
de blessures.

Le kitesurf est un sport de voile spécifique avec des lignes  qui ont
une longueur autour de  20-25 m. Ainsi, il nécessite de l'espace
et le moniteur pourra prendre un maximum de 4 ailes en l'air
pour dispenser un cours de qualité et en sécurité. 

\subsection{Diagnostic}
\begin{table}[h]
\centering
\begin{tabular}{|c|c|}
        \hline
        \textbf{Forces}                          & \textbf{Opportunités} \\ 
        \hline
        plusieurs sports sur le m\^eme site      &  outil d'intégration, meilleure ambiance\\
        possibilité d'évoluer dans l'association & plus de satisfaction client  \\
        avantages hébergement et nourriture      &                              \\
        \hline
        \textbf{Faiblesses}                      &  \textbf{Menaces} \\ 
        \hline
        salaires faibles                         & manque de cohésion d'équipe \\
        horaires décalés                         & concurrence, nouveaux acteurs   \\
        turn over                                &                               \\
        \hline
\end{tabular}
\caption{Analyse SWOT (Strengths, Weakeness, Opportunities, Threat)\label{swot}}
\end{table}
Le tableau \ref{swot} montre les quelques forces et faiblesses de la structure Ucpa.
Comme l'Ucpa est une association nationale, il est possible pour un jeune sportif
de se former et d'évoluer au sein de l'association, c'est une des grandes forces de l'Ucpa.
Parmi les faiblesses, pour un moniteur de kite, on peut noter des salaires plus
faibles que dans une structure privée. 
En menace potentielle, il y a  la concurrence avec 
l’apparition de nouveaux acteurs (par exemple décathlon travel), notamment
à l'internationale.
De plus, il  y a un turn over important parmi les équipes 
de moniteurs dans certains centres UCPA.

Une séance de kitesurf pour le personnel interne de l'Ucpa 
permet de renforcer la cohésion de l'équipe et aussi d'initier au kitesurf.
Ce sport n'est pas facilement  accessible du point de vue financier et
de l'encadrement.

Une après midi entre personnel interne de l'Ucpa à travers une 
activité sportive pourrait permettre  de mieux se connaître, 
en particulier, cette activité permet de partager en commun
un moment grisant dans un beau cadre naturel. De plus, l'idée
de faire une séance kite est faire progresser l'élève vers 
l'autonomie (voir appendice \ref{ucpa_projet}).

La zone de kite est assez restreinte mais bien adapté à l'apprentissage
du kitesurf avec du vent léger à modéré. De plus nous sommes en 
eau peu profonde, c'est à dire que le stagiaire à pied (environ 0.8 à 1.5 m
de profondeur) et sans vagues. Avec ces deux éléments, la zone de kite 
est propice à un apprentissage en sécurité.

En dehors de cette zone, la pratique de kitesurf sur le lac
y est interdite. De plus, le lac est en zone Natura 2000\cite{natura2000}, pour
protéger la biodiversité autour du lac.
Organiser un downwind ou une régate est donc au niveau
réglementaire compliqué. 

L'idée d'organiser une séance initiation kitesurf
au personnel de l'Ucpa d'Hourtin est donc venue logiquement.
Elle permettra au personnel de s'initier au kitesurf en 
sécurité, de participer à une activité plaisante
et de renforcer la cohésion des équipes en cohérence avec le 
projet éducatif et sportif de l'ucpa\ref{ucpa_projet}

Par ailleurs, les moniteurs de voile pourront mieux
comprendre les problématiques liés au kitesurf, notamment
la préparation du matériel, les  systèmes de sécurité 
(lâcher, libérer, larger), l'espace nécessaire, et la dérive des stagiaires.

Ainsi ils pourront se faire une idée des avantages et
inconvénients de ce sport vu sous l'angle de  l'enseignement.

Plusieurs séances ont été programmé et s'adressent au personnel 
interne de l'Ucpa (moniteurs et personnel administratif et restauration).
J'ai  proposé 12 places, mon collègue sylvain pourra prendre en charge 6 élèves.
Je présente dans ce rapport la première séance du 22 mai 2025 d'initiation 
au kitesurf. J'ai prévu une deuxième séance à la première semaine de septembre avec
pour thème une séance pour avancé.
L'intér\^et d'organiser une initiation kitesurf est de faire découvrir ce
sport aux moniteurs et de favoriser des échanges entre personnel dans 
une bonne ambiance.

Cela permettra aussi de montrer les avantages
et inconvénients de ce sport comme pratique nautique. 
Pour les moniteurs de voile en particulier, cet échange permet
de leur montrer les problématiques spécifiques du kite, 
comme la gestion de l'espace et la sécurité liée aux lignes du kite.

\subsection{Rétroplanning}
\begin{figure}[h]
\centering
%\includegraphics[width=12cm]{planningBL.png} 
\includegraphics[width=8cm]{Images/projet_planning2.jpg} 
\caption{Tableau de planification du projet \label{gantt}}
\end{figure}
Le planning des étapes importantes de mon projet est indiqué
à la figure \ref{gantt}. En mai, la décision de  réorienter 
le projet avec le changement de public a été décidé avec
ma responsable. 

J'ai organisé deux séances d'initiation kitesurf à destination des
débutants le 22 et 23 mai. La première semaine de septembre, j'ai 
aussi organisé une séance de kitesurf à destination des initiés.
Je présente ici dans ce rapport la première séance du 22 mai et la
séance pour les initiés.

\subsection{Estimation des co\^uts et du seuil de rentabilité}

\begin{figure}
\centering
\includegraphics[width=6cm]{Images/bateaux.jpg} 
\caption{Semi rigide utilisés pour les séances de kite\label{bateaux}}
\end{figure}
Pour avoir un ordre d'idée du coût d'une séance initiation kite, je
tiens compte du co\^ut du matériel et du co\^ut salarial. On rapporte 
cela au nombre de journée d'ouverture du centre.

Je suis en contrat pro avec l'UCPA. Le coût pour l'employeur (brut + charges) est
de 2089 euros pour 152 heures mensuel, 
ce qui implique un coût horaire de 14 euros pour l'employeur.
Pour mon tuteur, le coût horaire est de 19 euros.
Le prix hors taxe des équipements ainsi
que le prix pour la location du moteur des bateaux est indiqué
au tableau \ref{couts}.

\begin{figure}
\includegraphics[width=\linewidth]{Images/comptes.jpg} 
\caption{Calcul du prix de rentabilité en tenant compte
des co\^uts et de l'amortissement du matériel pour une séance de kite avec 3 ailes, 1 bateau et 2 moniteurs\label{couts}}
\end{figure}

Le budget prévisionnel permet d'estimer précisément les co\^uts et les recettes.
Ici, je liste tous les coûts impliqués dans la séance d’initiation kite, 
c'est  à dire le matériel avec les kites, barres, combinaisons ainsi que
la location du moteur du bateau. Il y a aussi le salaire horaire moniteur. 
On calcul le coût pour l'Ucpa
d'une séance de kite avec 3 kites (pour 3 binômes) à la journée.
Le centre de Hourtin est ouvert du 12/05/2025 au 01/09/2025 soit 15 semaines. 
Les stages kitesurf commencent en été, donc en realité, les kites servent
8 semaines, en comptant mai et juin, on peut arrondir à 10 semaines, soit 50 jours
par an.

Le coût total pour l'employeur pour 3h de séance avec le tuteur moniteur
et l'apprenti moniteur est de $3\times(14+19) = 3\times33 = 99 $.
Le coût total matériel est de 2600 euros, soit 52 euros par jour  en prenant
le bateau Myla pour 1 moniteur ($2600/50$).
Le coût de l'essence pour 5 l consommée, est de $1.8\times5 = 9$ euros .
Le coût de la journée est de $99 + 52 + 9 =  160 $ euros.
Le résumé des coûts journalier est indiqué au tableau \ref{cout_journalier}.

Je compte le matériel utilisé pour la séance d’initiation,
moi m\^eme et mon tuteur pour la partie ressources humaines.
Ainsi dans ce contexte,  pour 6 élèves, le seuil de rentabilité
d'un stage serait de $160/6 = 27$ euros
par jour. En dessous, l'Ucpa perdrait de l'argent.
On pourrait aussi faire cette exercice de budget prévisionnel avec
l'ensemble du matériel et des moniteurs de kite.

\begin{table}
\begin{centering}
\begin{tabular}{|c|c|c|}
\hline
\textbf{Détails}         & \textbf{Dépenses} & \textbf{Recettes}   \\
\hline
Matériel  & 52     &                     \\
\hline                                    
Essence   & 9      &                     \\
\hline              
Moniteurs  & 99    &                     \\
\hline
Total     & 160    &       160           \\
\hline
\end{tabular}
\caption{Bilan des co\^uts d'une séance kite à la journée\label{cout_journalier}}
\end{centering}
\end{table}

Cet exercice de lister tous les co\^uts permet de connaître le seuil 
de rentabilité et d'aider à calculer le prix des stages.

\section{Réalisation}
Plusieurs séances de kitesurf seront proposées aux moniteurs d'Hourtin.
Les dates sont le 22, 23 mai et le 2 et 3 juin.
Je présente ici dans ce projet la séance du 22 mai 2025.
Une séance perfectionnement sera organisée début septembre
et sera présenté dans ce rapport.

A l'UCPA, les groupes sont souvent de 6 élèves, nous
utilisons donc beaucoup le fonctionnement en bin\^ome. 
Lorsque le stagiaire a une bonne maîtrise de l'aile et
à l'étape du waterstart, nous pouvons commencer à fournir
un kite par élève.
\subsection{Objectifs}
Cette séance a pour objectifs d’initier au pilotage d’un kite et
de comprendre le fonctionnement de la fenêtre de vol. 
Pour les moniteurs de voile, elle vise également à 
sensibiliser aux contraintes spécifiques du kitesurf, 
notamment en matière d’espace et de sécurité.
 
Enfin, elle participe à la cohésion des équipes
dans le cadre d'une pratique sportive.

\begin{figure}
\centering
\includegraphics[width=10cm]{Images/hourtin_wind_220625.jpg} 
\caption{Météo seance 1, du 22/05/25\label{meteo}}
\end{figure}

\subsection{Séance initiation, 22 mai 2025}
%public: Sardine, Aurélien, Nais, Joe, Anthony, laureleen, Tristan, Rémi
%themes: pilotage, pilotage 1 main, nage tractée, nage tractée orientée, début waterstart

Le nom des stagiaires, leur \^age, leur poids, niveau et objectifs sont
indiqués dans le tableau \ref{stagiaires_table}.
Pour ma séance, je me suis occupé de Antony, Rémi, Laurette et Ewan.
Sylvain avait Aurélien (autonome), Josué (intermédiaire, tire des bords)
et Nais (waterstart et premiers bords).
Laureleen était absente.
Le vent était de Nord Ouest entre 10 et 14 noeuds comme indiqué sur
l'image \ref{meteo}.


\begin{table}
\begin{tabular}{|c|c|c|c|c|}
        \hline
        \textbf{Nom} & \textbf{Age} & \textbf{Poids}& \textbf{Niveau}     &  \textbf{Objectif} \\ 
        \hline
        Aurélien      &  34          &  72           &   Perfectionnement  & Reprise \\
        Nais          &  25          &  62           &   Débutant          & Comprendre et tirer des bords \\
        Josué         &  25          &  75           &   Intermédiaire     & Tirer des bords \\
        Antony        &  40          &  71           &   Débutant          & Utiliser la planche  \\
        Laureleen     &  -           &  -            &   -                 &   -  \\
        Tristan       &  35          & 82            &  Débutant           & Tirer des bords  \\
        Rémi          &  25          & 75            &  Débutant           &  Décoller  \\
        Laurette      &  29          & 54            &  Débutant           & Prendre du plaisir \\
        Ewan          &  22          & 75            & Débutant            &  -  \\
        \hline
\end{tabular}
\caption{Stagiaire de l'Ucpa pour ma séance initiation kitesurf du 22 mai 2025\label{stagiaires_table}}
\end{table}
Il est toujours instructif de connaître ses stagiaires, si ils sont 
sportifs ou non, si ils ont des blessures ou des appréhensions particulières.

Dans mon groupe, il y a Antony, un moniteur multi activité qui a un passé 
de sportif en haut niveau en rugby. Il est calme et attentif.
Tristan est un moniteur de surf, qui lui aussi est calme et attentif.
Ewan qui travaille en cuisine est un jeune sportif à l'écoute, qui avait 
fait déjà un peu de pilotage il y a 2 ans.
Enfin, il y a Laurette qui travaille à l'Ucpa en gestion et organisation, elle
est aussi attentive.

Pour le choix du la taille d'aile, on se base principalement sur le niveau du rider 
et son poids. J'avais donc équipé Antony et Tristan en $11 m^2$, Ewan qui avait
déjà un peu piloté il y a 2 ans, en $11 m^2$, Laurette était aussi en $11 m^2$.
Avec du recul, j'aurais du lui donner une  $9 m^2$, elle aurait été un peu plus
à l'aise dans le pilotage. Mais dans les prévisions météo, le vent n'était pas
indiqué à la hausse.


%\begin{itemize}
%\item Antony: Moniteur multi activité, passé de sportif en rugby. Calme et attentif.
%\item Tristan: Moniteur de surf, calme et attentif.
%\item Ewan: Travail en cuisine, sportif et à l'écoute
%\item Laurette: Travail à l'ucpa en gestion et organisation, attentive.
%\end{itemize}
\begin{figure}
\begin{minipage}{0.4\textwidth}
\includegraphics[width=8cm]{Images/imges_seance_1_2205/hourtin_wind_2.jpg} 
\caption{Un stagiaire qui à l'air heureux}
\end{minipage}
\hfill
\begin{minipage}{0.4\textwidth}
\includegraphics[width=8cm]{Images/imges_seance_1_2205/hourtin_wind_3.jpg} 
\caption{Une stagiaire concentrée sur le pilotage}
\end{minipage}
\end{figure}

\begin{figure}
\includegraphics[width=8cm]{Images/imges_seance_1_2205/hourtin_wind_6.jpg}
\caption{Stagiaires en bin\^ome} 
\end{figure}

\begin{table}
\begin{tabular}{|c|c|c|}
\hline
\textbf{Exercices}     &  \textbf{Repères}      \\
\hline 
Huits autour du zenith & Action droite, gauche   \\
\hline
Petits loops  & Action franche d'un coté  \\
\hline 
Aile stable à 11h30 ou 13h & Pilotage fin \\
\hline
Aile stable               & Pilotage à 1 main \\
\hline 
Aile stable               & Pilotage à 1 main en marchant \\
\hline
Aile en mouvement, 8      & Nage tractée en descendant le vent \\
\hline 
Aile stable à 10h30 ou 13h30   	& Nage tractée orientée, travers au vent \\
\hline
Aile au zenith, chausser la planche  & Planche perpendiculaire aux lignes, jambes fléchies \\
\hline
Waterstart                           &  Accéleration de l'aile, sortie les fesses de l'eau \\
\hline
\end{tabular}
\caption{Exercices pilotage jusqu'au waterstart\label{seance_pilotage}}
\end{table}

Mes stagiaires étaient tous des débutants (sauf Ewan, un faux débutant), 
le thème de la séance était d'apprendre le pilotage pour pouvoir contrôler son kite.
Les exercices de pilotage de difficulté croissante visant à une
meilleur maîtrise de l'aile, sont un pilotage à 2 mains en effectuant des huits
autour du zenith. Le but est de maintenir l'aile en l'air en contrôle.
Ensuite, le deuxième exercice est de pouvoir stabiliser le kite à différente 
zone, par exemple 11h, puis 12h, puis 13h.

Ensuite première nage tractée avec descente sous 
le vent. Pour cet exercice, il s'agit de se mettre à genoux
et de générer de la puissance avec le kite dans le but
de se faire tracter en avant.  L'idée est de se familiariser
avec le dosage de puissance du kite en le faisant plonger
plus ou moins vite dans la zone de puissance.

Le 3ème exercice est un pilotage à une main, et 
de maintenir le kite à une position stable autour
du zenith. On demande aussi au stagiaire de
détacher progressivement son regard de l'aile.
Cet exercice est important pour avoir une main libre
qui servira à prendre la planche ultérieurement.

Le 4ème exercice est la nage tractée orienté.
Le kite doit etre place à 10h30 ou 13h30, le pilotage
est à une main. Le stagiaire faire  à l'eau
avec son corps bien gainé en nage indienne. Le bras
libre est bien droit pointant dans une direction 
travers au vent. La nage tractée orientée est assez
technique, mais nécessaire à maîtriser pour récupérer 
sa planche.

Une fois ces étapes validées, on peut
amener une planche et essayer de chausser la planche
et d’être en contrôle du kite et de la planche, 
en résistance dans l'eau, jambes bien fléchies.

L'étape suivante est une sortie du corps de l'eau
en envoyant son aile en zone de puissance et tout en 
gardant les pieds dans les straps. 

L'étape suivante est de partir en glisse avec
une aile stabilisée et une trajectoire constante.

Ma fiche séance est indiquée en annexe \ref{fiche_seance}
et le résumé des exercices de pilotage avec pour
finalité un bon controle de l'aile est donné
au tableau \ref{seance_pilotage}.

\subsection{Déroulé de la séance}
Le but de la séance est d'apprendre à faire voler le kite et
apprendre le pilotage du kite avec sa fen\^etre de vol.
\begin{figure}
\centering
\includegraphics[width=8cm]{Images/wind-window-kiteboarding.png}
\caption{fen\^etre de vol d'un kite\label{fenetre}}
\end{figure} 

Après \^etre arrivé sur zone, nous gonflons le premier kite avec les
lignes préalablement connectées à terre. J'effectue une petite démo
de pilotage, puis fait passer tout le monde chacun son tour pour
un premier pilotage.
Ensuite, je vais passer le kite au binôme Tristan et Antony après leur
avoir expliqué le redécollage. Mon formateur Romain est présent pendant la
séance et il prodigue quelques conseils supplémentaires à Tristan et Antony.
Il me donne aussi quelques astuces et conseils pour le bon déroulé de
la séance.
Je gonfle le 2ème kite pour le binome Ewan et Laurette.
Nous restons 20 à 30 min pour travailler les premiers pilotages, donner des conseils 
et corriger des erreurs.
Ensuite, voyant que Ewan se débrouillait très bien avec un bon contrôle du 
kite, j'ai décidé de gréer un 3ème kite pour lui.
Il y avait donc pour le reste de la séance, 1 kite pour Tristan et Antony.
1 kite pour Ewan, et 1 kite pour Laurette. Je suis resté pas mal de temps
avec Laurette pour la coacher, car elle avait des difficultés à contrôler l'aile.
J'étais donc le plus souvent avec Laurette et Ewan, pour les corriger techniquement 
et aussi pour effectuer quelques démos.
En fin de séance, j'ai expliqué le waterstart à Ewan et je lui ai
amené une planche pour qu'il puisse effectuer ses premières sorties
de l'eau avec la planche.
J'ai bien expliqué aux stagiaires que piloter dans le vent léger est
formateur et plus technique que dans du vent établi. La raison est qu'il faut
piloter le kite dynamiquement pour créer plus de vent vitesse pour le kite.
De plus la fen\^etre se réduit, pour que le kite vol dans du vent léger, 
il faut par exemple le faire évoluer entre 11h et 13h. Si le kite s'approche du 
bord de fen\^etre, parfois un loop pour qu'il garde de la vitesse et le replacer
en zone de puissance est la meilleure solution.
Pour les loops dans le vent léger, j'indique aussi qu'il faut
des commandes franches (gros différentiel) pour effectuer un petit loop.
Ce point pour les stagiaires peu sembler contre intuitif.

Au final, la séance s'est bien passé et les stagiaires étaient content.
Après avoir dégonflé les kites sur le bateaux, nous avons un fait un
petit débrief de la séance.

\subsection{Séance 2 perfectionnement}
à compléter
\section{Evaluation}
Le retour des stagiaires, selon le questionnaire de l'annexe \ref{questionnaire}.
Ils ont tous aimé l'activité kitesurf et se sont sentis en sécurité.
Une stagiaire s'est sentie trop crispée et avait peut \^etre un peu d'appréhension.
Ils  voudraient bien continuer à progresser si l'opportunité se représentait.
Les réponses de chaque stagiaires sont consignées dans à l'annexe \ref{reponses}.

Les stagiaires ont aussi compris le fonctionnement d'un kite avec le pilotage.
Ils souhaitent tous en refaire si ils en ont la possibilité.

En ce qui concerne les points positifs et négatifs, on peut noter
que plus d'explication théorique pour Ewan serait utile, pour Antony
il souhaite moins d'explication des termes techniques, pour plus de clarté.
Pour Tristan, un petit topo théorique avant la mise en place serait utile.

Les axes d'améliorations sont peut \^etre de faire participer
plus les stagiaires à la mise en place du matériel, et de
donner des consignes simples et claires. 

En résumé, les stagiaires ont apprécié l'initiation kitesurf et
souhaiteraient renouveler l'expérience pour progresser.

\section{Conclusion}

Pendant ma saison à l'UCPA d'Hourtin, j'ai apprécié proposer une 
séance d'initiation kitesurf à destination du personnel UCPA.
Ce projet a permis de renforcer la cohésion d'équipe au sein de l'UCPA
tout en initiant le personnel aux bases du kitesurf. 

Ma séance s'est bien déroulée en sécurité et en optimisant 
le temps de pratique. Les stagiaires ont été très satisfait de la 
séance et la plupart voudraient continuer l'activité si ils en 
ont la possibilité. 
Au final cette initiation kitesurf s'est révélée très positive.
Il reste un regret, celui de ne pas pu avoir réaliser mon idée initiale, 
à destination d'un public différent. Je pense que la clé pour
mettre en place une séance avec tel public est d'avoir un contact
direct et motivé dans une de ces structures.

Aparté: Je suis en train d'écrire ce rapport fin juin 2025 et L’Europe subie
une canicule précoce avec des températures de $40^{\circ}$ C en France. Ici à Hourtin, 
il fait $34^{\circ}$ C.\footnote{
La planète terre est en train de franchir le seuil  de  $+\, 1.5^{\circ}$ C par rapport
 à l'ère pré industrielle. La consommation de pétrole et de gaz ne baissent pas, les 
émissions de C02 continuent d'augmenter\cite{giec}}.
Je m'interroge sur les métiers d'éducateurs plein air dans le futur, il sera
probablement plus possible d'exercer en été certains jours
à cause de la chaleur excessive.

\appendix
\appendixpage
\addappheadtotoc
\chapter{Questionnaire de satisfaction\label{questionnaire}}
\begin{itemize}
\item 1. As tu aimé l'activité kitesurf ?
\item 2. Te sentais tu en sécurité ?, as tu eu peur à un moment ?
\item 3. As tu compris des choses sur le kitesurf ?
\item 4. Aimerais tu continuer cette activité si tu en as la possibilité.
\item 5. As des points positifs ou négatifs a mentionner ?
\item 6. D'autres commentaires ?
\end{itemize}

\chapter{Réponses Stagiaires\label{reponses}}

\begin{itemize}
\item 1. \textbf{Ewan}: Oui, c'était super!, \textbf{Antony}: Oui, j'ai
beaucoup aimé les sensations du vent  dans la voile et du début de
pilotage que nous avons  pu réaliser.                                      
\textbf{Laurette}: Oui, c'était très sympa!, \textbf{Tristan}: Oui,
c'était un moment plaisant avec de bonnes sensations, m\^eme pour
une première séance.
\item 2. \textbf{Ewan}: Non pas du tout, les consignes de sécurité
ont été bien expliquées, donc je me suis senti en total sécurité,
\textbf{Antony}: Je n'ai pas eu peur et je me suis senti en
sécurité gr\^ace aux conseils et à l'aspect ludique donnée à la séance 
notamment avec la nage tractée, \textbf{Laurette}: Je me suis
toujours dit que jamais je ne ferai du kitesurf (peur de m'envoler),
bon c'est ce qui s'est un peu passé! Les explications étaient 
claires et l'activité bien encadrée, je me suis sentie en sécurité.
Par contre, j'étais clairement trop crispée et j'ai peut \^etre
oublié de respirer. Vive les crampes en fin de séance ! 
\item 3. \textbf{Ewan}: Le pilotage et la nage tracté sont venus
assez rapidement mais il faudrait encore davantage de pratique
pour vraiment assimiler et maîtriser le tout à 100\%
\textbf{Antony}: J'ai bien compris les premières notions de
pilotage et la logique \textbf{Laurette}: Oui j'ai compris
toutes les explications !, les mettre en pratique est un
peu plus compliqué ...
\textbf{Tristan}: J'ai compris quelques bases, je suppose
notamment sur la préparation du matériel et le décollage de l'aile
\item 4. \textbf{Ewan}: Oui!, \textbf{Antony}: Oui, j'espère
avoir la possibilité d'en refaire afin de pouvoir progresser
dans l'activité, \textbf{Laurette}: Ce serait avec grand plaisir
de renouveler l'expérience.
\textbf{Tristan}: Absolument!, 
\item 5. \textbf{Ewan}: En terme pédagogique je pense qu'il
pourrait \^etre bien d'utiliser des images ou métaphores pour
aider à mieux comprendre les différentes techniques (exemple
du boxeur pour maîtriser les gauches droites sur le  pilotage).
Et peut \^etre prendre plus de temps sur les termes techniques
qui sont propre à cette discipline (les arrières, la fen\^etre, ect...) 
Sinon, l'activité était super, merci beaucoup !
\textbf{Antony}: Peut \^etre aborder les termes techniques un 
peu plus tard ou après une définition des termes car j'étais un
peu perdu. L'aspect ludique de la séance et le fait que l'on
soit en bin\^ome \textbf{Laurette}: Positifs: moniteurs à
l'écoute, négatifs: j'aurais bien aimé essayer avec une
voile un peu plus adaptée à mon gabarit.\textbf{Tristan}: 
Pas négatif, mais j'imaginerai avoir un cours théorique rapide avant
la mise en pratique afin de comprendre les bases du kitesurf
\item 6 \textbf{Ewan}: ras, \textbf{Antony}: ras, 
\textbf{Laurette}: simplement merci!, \textbf{Tristan}: 
Pas spécialement, la question précédente suffira de mon coté:)
\end{itemize}


\chapter{Echanges de mail envoyé à une association Aria33}\label{appendix_mail}
\textbf{Cette annexe contient les échanges de mail que j'ai eu avec l'association 
en gironde Aria33\cite{aria33}} \\

Bonjour,

Je suis actuellement en formation BP Kitesurf à l'école nationale de Quiberon.
Dans le cadre de notre formation, nous devons réaliser un projet dans
une structure en lien avec l'enseignement du kitesurf.

Ma structure d'accueil sera l'UCPA à Hourtin.
Mon idée de projet est de proposer une journée initiation kitesurf à
des jeunes qui ont eu un parcours difficile pour leur faire découvrir
un nouveau sport.
(pour la tranche d'age, entre 10 et 18 ans par exemple).

Je me permets de vous contacter pour savoir si il y aurait des
associations dans la région (proche d'Hourtin) qui pourrait etre
intéressées par mon projet.

Je ne sais pas encore exactement le nombre, probablement entre 6 et 8.
Comme il y a des stages en semaine à l'UCPA, je pensais organiser sur
le samedi, dimanche ou les deux. Les mois possibles seront mai ou
juin.

De mon coté je dois encore vérifier si mon idée est validée par mon
tuteur. (Je suis en contact).


Bien cordialement

Benjamin Lepers

\bigskip
Bonjour
Merci pour cette estimation
En effet, les jeunes que nous accompagnons sont en situation de grande
précarité pour nombre d'entre eux. Si une baisse maximale du prix
pouvait être proposée, nous serions probablement dans la capacité
de constituer un groupe de 6 jeunes. Sur un samedi.

Cordialement
Dolly LEBON
Référente ARIA33

\bigskip
Bonjour,

j'ai contacté ma tutrice de l'Ucpa d'hourtin pour le budget.
Pour une séance de kite  pour 6 élèves, le budget serait de 283 euros,
soit 47 euros par personnes.

Si ça bloque peut être qu'elle pourrait baisser un peu.


Je reste disponible.

Cordialement
\bigskip
Bonjour,

j'ai eu le retour de ma responsable de la base d'Hourtin, et l'UCPA
pourrait faire une baisse de 20\,\%, soit 226 euros pour 6, soit 37 euros
par élève.

Cordialement

Benjamin Lepers
\chapter{Fiche séance\label{fiche_seance}}
\section{Mise en place}
\begin{itemize}
\item 6 élèves adultes
\item matériel nécessaire: 6 kites, 7,7,9,9,12,12, 3 Twin Tip grandes
\item moniteur: gilet, harnais, coupe ligne, radio, trousse de secours, couteau, coupe ligne
\item bateau: lomac 115 ch
\item météo: 
\end{itemize}
\section{Consignes de sécurité}
\begin{itemize}
\item rappel des 3 sécurités: lâcher, déclencher, libérer
\item distance entre stagiaires: minimum 2 longueurs de lignes
\item si l'aile tombe en feuille morte, on met la main sur le largeur au cas ou elle redécolle et tire
\end{itemize}
\section{Communication}
\begin{itemize}
\item signes: aile au zénith, aile en parking à droite ou gauche, lacher la barre
\item sifflement: on se replace dans la zone
\end{itemize}
\section{Déroulé de la séance}
On commence par une aile en l'air pilotée par le moniteur.
Les stagiaires à proximité écoutent les conseils de pilotage et les 3 systèmes
de sécurité.
Ensuite, l'aile est passée à tour de rôle pour un premier pilotage.
Lorsque, le redécollage est maîtrisé et les consignes respectées, 
on gonfle une 2ème aile, il y a maintenant 3 personnes par kite.
Si ca tourne bien en sécurité, on gonfle un 3ème kite et on a 3 binôme.


\chapter{Projet UCPA\label{ucpa_projet}}
%\textbf{Ceci est la brochure de l'UCPA décrivant son projet éducatif
%pour les stagiaires}

\includepdf[angle=90]{projet_ucpa2025.pdf}
\bibliographystyle{plain}
\bibliography{biblio}
\end{document}